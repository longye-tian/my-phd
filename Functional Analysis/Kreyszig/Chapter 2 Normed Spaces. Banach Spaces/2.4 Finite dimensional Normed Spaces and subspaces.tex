\section{Finite dimensional normed space and subspaces}

\begin{frame}{Linear Combination Lemma}
Let $\{x_1, x_2,\cdots,x_n\}$ be a linearly independent set of vectors in a normed space $X$ of any dimension. Then there is a number $c>0$ such that for every choice of scalars $a_1,\cdots, a_n$, we have,
$$
\|a_1x_1+\cdots +a_nx_n\|\ge c(|a_1|+\cdots|a_n|)
$$
    
\end{frame}

\begin{frame}{Completeness}
Every finite dimensional subspace $Y$ of a normed space $X$ is complete. Every finite dimensional normed space $X$ is complete.
\end{frame}

\begin{frame}{Closedness}
Every finite dimensional subspace $Y$ of a normed space $X$ is closed.
    
\end{frame}

\begin{frame}{Equivalence norm}
A norm $\|\cdot\|$ on a vector space $X$ is said to be equivalent to a norm $\|\cdot\|_0$ on $X$ if there are positive numbers $a$ and $b$ such that for all $x\in X$ we have
$$
a\|x\|_0\le \|x\|\le b\|x\|_0
$$

\begin{itemize}
    \item Equivalent norms on $X$ defines the same topology for $X$
    \item Every norms are equivalent on finite dimensional normed space.
    \item Every Cauchy sequence is the same equivalent normed space
    \item Convergence is the same on equivalent normed space 
\end{itemize}
    
\end{frame}