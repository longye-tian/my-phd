%%%%%%%%%%%%%%%%%%%%%%%%%%%%%%%%%%%%%%%%%
% Professional Mathematical Presentation Template
% 
% This template uses the beamer class with the Madrid theme
% and a custom color scheme for a clean, professional look
% that works well with mathematical content.
%%%%%%%%%%%%%%%%%%%%%%%%%%%%%%%%%%%%%%%%%

\documentclass[aspectratio=169]{beamer} % 16:9 aspect ratio (modern)

% Theme settings
\usetheme{Madrid}
\usecolortheme{default}


\definecolor{primcolor}{RGB}{25,50,100} % Dark blue
\setbeamercolor{structure}{fg=primcolor}
\setbeamercolor{frametitle}{bg=primcolor!15, fg=primcolor}
\setbeamercolor{title}{fg=white} % White title text for contrast
\setbeamercolor{subtitle}{fg=white} % White subtitle text
\setbeamercolor{author}{fg=primcolor} % White author text
\setbeamercolor{date}{fg=primcolor} % White date text
\setbeamercolor{institute}{fg=primcolor} % White institute text

% Font settings
\usefonttheme{professionalfonts}
\usefonttheme{serif}

% Package imports
\usepackage{amsmath, amsfonts, amssymb, amsthm} % Math packages
\usepackage{mathtools} % Enhanced math tools
\usepackage{bm} % Bold math symbols
\usepackage{graphicx} % For images
\usepackage{booktabs} % Professional tables
\usepackage{tikz} % For diagrams
\usetikzlibrary{arrows, positioning, matrix, decorations.pathreplacing}

% Use beamer's theorem styles
\setbeamertemplate{theorem}[ams style]
\setbeamertemplate{theorems}[numbered]


% Remove navigation symbols
\setbeamertemplate{navigation symbols}{}

% Custom footer
\setbeamertemplate{footline}{
  \leavevmode%
  \hbox{%
  \begin{beamercolorbox}[wd=.333333\paperwidth,ht=2.25ex,dp=1ex,center]{author in head/foot}%
    \usebeamerfont{author in head/foot}\insertshortauthor
  \end{beamercolorbox}%
  \begin{beamercolorbox}[wd=.333333\paperwidth,ht=2.25ex,dp=1ex,center]{title in head/foot}%
    \usebeamerfont{title in head/foot}\insertshorttitle
  \end{beamercolorbox}%
  \begin{beamercolorbox}[wd=.333333\paperwidth,ht=2.25ex,dp=1ex,right]{date in head/foot}%
    \usebeamerfont{date in head/foot}\insertshortdate{}\hspace*{2em}
    \insertframenumber{} / \inserttotalframenumber\hspace*{2ex} 
  \end{beamercolorbox}}%
  \vskip0pt%
}

% Title information
\title[DL]{Use Deep Learning to Solve RBC model}
\subtitle{A cookbook}
\author[Longye]{Longye Tian \\ \texttt{longye.tian@anu.edu.au}}
\institute[ANU]{Australian National University\\ School of Economics}
\date{April 10th, 2025}
\DeclareFontFamily{U}{mathx}{\hyphenchar\font45}
\DeclareFontShape{U}{mathx}{m}{n}{
      <5> <6> <7> <8> <9> <10>
      <10.95> <12> <14.4> <17.28> <20.74> <24.88>
      mathx10
      }{}
\DeclareSymbolFont{mathx}{U}{mathx}{m}{n}
\DeclareMathSymbol{\bigtimes}{1}{mathx}{"91}

\begin{document}

% Title frame
\begin{frame}
  \titlepage
\end{frame}

% Outline frame
\begin{frame}{Outline}
  \begin{enumerate}
      \item General Cookbook procedure
      \item RBC model as example
  \end{enumerate}
\end{frame}

\begin{frame}{Procedures}
\begin{enumerate}
    \item Transform a DSGE model into a DL problem
    \begin{enumerate}
        \item First-order/equilibrium conditions
        \item Establish the Loss function
        \item AIO operator trick
    \end{enumerate}
    \item Solve the DL problem
    \begin{enumerate}
        \item Neural network to approximate policy function
        \item Gradient Descent to minimize loss function
        \item Evaluate the result
    \end{enumerate}
\end{enumerate}
    
\end{frame}


\begin{frame}{RBC model - Representative Consumer}
Representative Consumer chooses consumption $c_t$, capital $k_{t+1}$, labor $h_t$, investment $x_t$ to maximize expected lifetime utility:
$$ \max_{\{c_t, k_{t+1}, h_t,x_t\}_{t=0}^{\infty}} E \big[ \beta^t \Big(\eta \log(c_t) + (1-\eta)\log(1 - h_t) \Big) \Big]$$
subject to the budget constraint and capital's law of motion:

\begin{align*}
    c_t + k_{t+1} &= y_t + (1 - \delta)k_{t}\tag{1}\\
    k_{t+1} &= (1-\delta)k_t + x_t\tag{2}
\end{align*}
   
    
\end{frame}

\begin{frame}{RBC model - Representative Firm}
    A representative firm combines capital $k_t$ and labor $h_t$ to create $y_t$ units of a consumption good using a Cobb-Douglas production function: 
    \begin{align*}
        y_t = a_t k_t^{\alpha} h_t^{1-\alpha}\tag{3}
    \end{align*}

The logarithm of total factor productivity (TFP), where TFP is denoted by $a_t$, follows an AR(1) process
\begin{align*}
    \log(a_t) = \rho  \log(a_{t-1}) + \varepsilon_t\tag{4}
\end{align*}
where $\varepsilon\sim\mathcal{N}(0,\sigma_\epsilon^2)$.
\end{frame}
\begin{frame}{First-order conditions}
FOC for consumption and capital (Euler equation)
\begin{align*}
    1 = \beta E_t \Big[ \frac{c_t}{c_{t+1}}\big(\alpha \frac{y_{t+1}}{k_t} + 1 -\delta \big) \Big]\tag{5}
\end{align*}
FOC for labor
\begin{align*}
    (1-\eta) c_t h_t = \eta(1- \alpha)(1 - h_t) y_t \tag{6}
\end{align*}
    
\end{frame}

\begin{frame}{Dynamic equilibrium characterization: 6 equations, 6 unknown}
Six unknown: $c_t, k_{t+1}, y_t, h_t, x_t, a_t$
\begin{align*}
     c_t + k_{t+1} &= y_t + (1 - \delta)k_{t}\tag{Budget Constr.}\\
    k_{t+1} &= (1-\delta)k_t + x_t\tag{Capital LoM}\\
     y_t &= a_t k_t^{\alpha} h_t^{1-\alpha}\tag{Production}\\
     \log(a_t) &= \rho  \log(a_{t-1}) + \varepsilon_t\tag{Exog. TFP}\\
     1 &= \beta E_t \Big[ \frac{c_t}{c_{t+1}}\big(\alpha \frac{y_{t+1}}{k_t} + 1 -\delta \big) \Big]\tag{Euler eq.}\\
     (1-\eta) c_t h_t &= \eta(1- \alpha)(1 - h_t) y_t \tag{Labor foc}\\
\end{align*}
Parameters:$\alpha, \beta, \delta, \eta, \rho,\sigma_\epsilon$
\end{frame}


\end{document}
