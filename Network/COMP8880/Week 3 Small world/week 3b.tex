\documentclass[11pt,a4paper]{article}
\usepackage{amsmath,amssymb,amsthm}
\usepackage{graphicx}
\usepackage{geometry}
\usepackage{enumitem}
\geometry{margin=2cm}
\usepackage{hyperref}
\theoremstyle{definition}
\newtheorem{definition}{Definition}
\newtheorem{theorem}{Theorem}
\newtheorem{lemma}{Lemma}
\newtheorem{example}{Example}
\title{Small World Networks and Navigation: Exam Study Guide}
\author{COMP 4880/8880 Network Science}
\date{}
\begin{document}
\maketitle
\section{Introduction and Motivation (Pages 1-7)}
\subsection{The Small World Phenomenon}
The small world phenomenon refers to the surprising property that in many real-world networks, any two nodes can be connected by remarkably short paths. This concept gained prominence through:
\textbf{Milgram's Experiment (1967):} Stanley Milgram conducted an experiment where people in Nebraska tried to send letters to a target person in Boston through chains of acquaintances. Key findings:
\begin{itemize}
\item Average chain length was about 6 intermediaries ("six degrees of separation")
\item People successfully found short paths using only local information
\item Geographic distance progressively decreased as chains approached the target
\end{itemize}
\textbf{Real-World Examples:} Table 3.2 (page 5) shows various networks exhibit small world properties:
\begin{itemize}
\item Internet: $\langle d \rangle = 6.98$, $d_{max} = 26$
\item WWW: $\langle d \rangle = 11.27$, $d_{max} = 93$
\item Email networks: $\langle d \rangle = 5.88$, $d_{max} = 18$
\end{itemize}
The formula $\langle d \rangle \approx \frac{\ln N}{\ln\langle k \rangle}$ provides a reasonable approximation for average distance.
\subsection{The Paradox}
Real networks exhibit two seemingly contradictory properties:
\begin{itemize}
\item \textbf{High clustering:} Friends of friends tend to be friends (local structure)
\item \textbf{Low diameter:} Short paths exist between any two nodes (global property)
\end{itemize}
Random graphs (Erdős-Rényi) have low diameter but essentially no clustering. Regular lattices have high clustering but large diameter. How can we have both?
\section{The Watts-Strogatz Model (Pages 8-17)}
\subsection{Model Construction}
The Watts-Strogatz model creates small world networks through a simple process:
\begin{enumerate}
\item Start with a ring of $n$ nodes, each connected to $k$ nearest neighbors
\item With probability $p$, rewire each edge to a random node
\item Alternative: Add random edges with probability $p$ (preserving base structure)
\end{enumerate}
\subsection{Key Properties}
The model exhibits a phase transition as $p$ increases from 0 to 1:\\
\\
\textbf{Clustering Coefficient:}\\
$$
C(p) = C(0)\cdot (1-p)^3
$$
Where $C(0) = \frac{3(k-2)}{4(k-1)}$ for the initial ring. The clustering remains high even for small $p$ because rewiring affects only a small fraction of triangles.\\
\\
\textbf{Average Path Length:}\\
For small $p$, the path length drops dramatically while clustering remains high. This creates the "small world regime" where $0.001 < p < 0.1$.\\
\\
\textbf{Diameter Analysis (Alternative 2D Grid):}\\
Starting with a square grid where each node has one random long-range edge:
\begin{itemize}
\item Partition grid into $2 \times 2$ subgraphs (supernodes)
\item Each supernode has 4 outgoing long-range edges
\item Creates a 4-regular random graph on supernodes
\item Diameter is $O(\log n)$
\end{itemize}
\section{Network Navigation (Pages 18-24)}
\subsection{Decentralized Search Problem}
The key insight from Milgram's experiment: people can find short paths using only local information. This leads to the decentralized search problem:
\begin{definition}[Decentralized Search]
Given:
\begin{itemize}
\item Source node $s$ knows only its neighbors and target $t$'s location
\item No global knowledge of network structure
\item Geographic navigation: choose neighbor closest to target
\end{itemize}
Find: Path from $s$ to $t$
\end{definition}
\subsection{Navigation in Watts-Strogatz Networks}
\begin{theorem}
Decentralized search in Watts-Strogatz networks (with uniformly random long-range links) requires $\Omega(\sqrt{n})$ steps in expectation for 1D rings, and $\Omega(n^{2/3})$ steps for 2D grids.
\end{theorem}
\textbf{Proof Sketch (1D case):}
\begin{enumerate}
\item Define $K$ = interval of $\sqrt{n}$ nodes around target
\item Probability any node has link to $K$ is $\frac{2\sqrt{n}}{n} = \frac{2}{\sqrt{n}}$
\item Expected steps to reach $K$ is $\frac{\sqrt{n}}{2}$
\item Once in $K$, need $O(\sqrt{n})$ more steps
\item Total: $\Omega(\sqrt{n})$ steps
\end{enumerate}
This uses the \textbf{principle of deferred decisions}: we can assume random links are created only when nodes are visited.
\section{Kleinberg's Model (Pages 25-37)}
\subsection{Model Definition}
Kleinberg's key insight: geographic bias in long-range connections enables efficient navigation.
\begin{definition}[Kleinberg's Model]
\begin{itemize}
\item Nodes on a $d$-dimensional grid
\item Each node has local connections to nearest neighbors
\item One long-range connection with probability:
$P(u\rightarrow v)=d(u,v)−\alpha\sum_{w\neq u}d(u,w)−\alpha P(u \rightarrow v) = \frac{d(u,v)^{-\alpha}}{\sum_{w \neq u} d(u,w)^{-\alpha}}P(u\rightarrow v)=\sum_{w\neq u}-\alpha d(u,v)−\alpha$
where $d(u,v)$ is grid distance and $\alpha \geq 0$ is the parameter
\end{itemize}
\end{definition}
\subsection{Navigation Performance}
\begin{theorem}[Kleinberg 2000]
Decentralized search time in $d$-dimensional grids:
\begin{itemize}
\item $\alpha = d$: $O((\log n)^2)$ steps (efficient)
\item $\alpha \neq d$: $\Omega(n^{\beta})$ steps for some $\beta > 0$ (inefficient)
\end{itemize}
\end{theorem}
For the special case of 2D grids ($d=2$), only $\alpha = 2$ enables polylogarithmic search time.
\subsection{Why $\alpha = d$ Works}
The key insight is that $P(u \rightarrow v) \sim d(u,v)^{-d}$ creates a scale-invariant distribution:
\begin{itemize}
\item Number of nodes at distance $r$: $\Theta(r^{d-1})$ (surface area of sphere)
\item Probability of linking to each: $\Theta(r^{-d})$
\item Total probability to any node at distance $r$: $\Theta(r^{d-1}) \cdot \Theta(r^{-d}) = \Theta(1/r)$
\item This gives equal probability across all distance scales!
\end{itemize}
\section{Proof of Kleinberg's Navigation Theorem (Pages 29-33)}
\subsection{Upper Bound for $\alpha = 1$ in 1D}
We prove that greedy routing takes $O((\log n)^2)$ steps when $\alpha = 1$.
\textbf{Phase-based Analysis:}
\begin{itemize}
\item Divide search into phases: phase $j$ when distance to target is between $2^j$ and $2^{j+1}$
\item Total phases: $O(\log n)$
\item Goal: Show each phase takes $O(\log n)$ steps in expectation
\end{itemize}
\textbf{Key Steps:}

\textbf{Normalizing constant bound:}
$$
Z = \sum_{w\neq v} \frac{1}{d(v,w)}\le 2\sum_{k=1}^{n/2} \frac{1}{k}\le 2\ln n
$$

Therefore: $P(v \rightarrow w) \geq \frac{1}{\log n} \cdot \frac{1}{d(v,w)}$

\textbf{Progress probability:}
In phase $j$ (distance $\leq 2^j$ to target), probability of halving distance:
\begin{itemize}
\item Set $I$ = nodes within distance $2^{j-1}$ of target
\item $|I| \geq 2^{j-1}$ nodes
\item For any $w \in I$: $d(v,w) \leq \frac{3}{2} \cdot 2^j$
\item $P(v \rightarrow I) \geq \frac{2^{j-1}}{3 \cdot 2^j \log n} = \frac{1}{6\log n}$
\end{itemize}
\textbf{Phase length:}
Expected steps in phase $j$:
$$
\mathbb{E}[X_j]\le \frac{1}{1/(6\log n)}=6\log n
$$
\textbf{Total time:}
$$
\mathbb{E}[\text{Total steps}] = \sum_{j=1}^{\log n} \mathbb{E}[X_j]\le 6\log n\cdot \log n = \mathcal{O}((\log n)^2)
$$

\subsection{Extension to Higher Dimensions}
For $d$-dimensional grids with $\alpha = d$:
\begin{itemize}
\item Number of nodes at distance $r$: $\Theta(r^d)$
\item Probability density: $r^{-d}$
\item Similar analysis yields $O((\log n)^2)$ search time
\end{itemize}
\section{Lower Bounds and Impossibility Results (Pages 21, 36)}
\subsection{Why Other Values of $\alpha$ Fail}
\textbf{Case $\alpha < d$ (too many long links):}
\begin{itemize}
\item Long-range links overshoot the target
\item Cannot make consistent progress toward target
\item Search time: $\Omega(n^{(d-\alpha)/d})$
\end{itemize}
\textbf{Case $\alpha > d$ (too many short links):}
\begin{itemize}
\item Links are too local to make significant progress
\item Must traverse many intermediate regions
\item Search time: $\Omega(n^{(\alpha-d)/(d+1)})$
\end{itemize}
\section{Key Takeaways for the Exam}
\subsection{Conceptual Understanding}
\begin{enumerate}
\item Small world networks combine high clustering with short paths
\item Watts-Strogatz creates small worlds via random rewiring
\item Navigation requires both network structure AND search algorithm
\item Geographic information enables efficient decentralized search
\end{enumerate}
\subsection{Mathematical Results}
\begin{enumerate}
\item Watts-Strogatz diameter: $O(\log n)$
\item Navigation in Watts-Strogatz: $\Omega(\sqrt{n})$ (1D), $\Omega(n^{2/3})$ (2D)
\item Kleinberg optimal: $\alpha = d$ gives $O((\log n)^2)$ search
\item Scale-invariance at $\alpha = d$ is crucial for navigation
\end{enumerate}
\subsection{Proof Techniques}
\begin{enumerate}
\item Phase-based analysis for upper bounds
\item Principle of deferred decisions for random structures
\item Probabilistic arguments for expected values
\item Geometric series for summing over phases
\end{enumerate}
\textbf{Remember:} The key insight is that successful navigation requires the right balance between local structure (clustering) and long-range connections with appropriate distance-dependent probabilities.

\section*{Instructions}
This practice exam contains questions in the format you'll see on the actual exam: True/False, Multiple Choice, and Short Answer questions. Each question includes a detailed solution to help you prepare effectively.
\section{True/False Questions}
\subsection{Question T1}
\textbf{Statement:} In the Watts-Strogatz model with rewiring probability $p = 0.01$, the clustering coefficient drops to less than $10\%$ of its original value.\\
\\
\textbf{Answer:} False\\
\\
\textbf{Explanation:} The clustering coefficient in the Watts-Strogatz model follows $C(p) = C(0) \cdot (1-p)^3$.For $p = 0.01$:
$$
C(0.01) = C(0)\cdot (1-0.01)^3  = C(0)\cdot 0.99^3\approx C(0)\cdot 0.97
$$
So the clustering coefficient remains at approximately $97\%$ of its original value, not less than $10\%$.
\subsection{Question T2}
\textbf{Statement:} Milgram's experiment demonstrated that people can find short paths in social networks using only local information.\\
\\
\textbf{Answer:} True\\
\\
\textbf{Explanation:} This is the key finding of Milgram's experiment. Participants successfully routed letters from Nebraska to Boston with an average of 6 intermediaries, using only knowledge of their immediate contacts and the target's location.
\subsection{Question T3}
\textbf{Statement:} In Kleinberg's model, navigation is efficient for any value of the parameter $\alpha > 0$.\\
\\
\textbf{Answer:} False\\
\\
\textbf{Explanation:} Navigation is only efficient when $\alpha = d$ (where $d$ is the dimension of the grid). For $\alpha \neq d$, decentralized search requires polynomial time in $n$, not polylogarithmic time.
\subsection{Question T4}
\textbf{Statement:} The diameter of the Watts-Strogatz model is $O(\log n)$ even for very small rewiring probability $p$.\\
\\
\textbf{Answer:} True\\
\\
\textbf{Explanation:} Even a small number of random long-range edges creates shortcuts that reduce the diameter from $O(n)$ to $O(\log n)$. This is why the ``small world regime" exists for $p$ as small as $0.001$.
\section{Multiple Choice Questions}
\subsection{Question M1}
In a 2D grid with Kleinberg's model, what value of $\alpha$ enables efficient decentralized search?
\begin{enumerate}[label=(\alph*)]
\item $\alpha = 0$
\item $\alpha = 1$
\item $\alpha = 2$
\item $\alpha = 4$
\end{enumerate}
\textbf{Answer:} (c) $\alpha = 2$\\
\\
\textbf{Explanation:} For efficient navigation in Kleinberg's model, we need $\alpha = d$ where $d$ is the dimension. In a 2D grid, $d = 2$, so $\alpha = 2$ is optimal. This creates a scale-invariant distribution where the probability of having a link at any distance scale is roughly constant.
\subsection{Question M2}
What is the expected search time for decentralized navigation in a 1D Watts-Strogatz network with $n$ nodes?
\begin{enumerate}[label=(\alph*)]
\item $O(\log n)$
\item $O((\log n)^2)$
\item $O(\sqrt{n})$
\item $O(n)$
\end{enumerate}
\textbf{Answer:} (c) $O(\sqrt{n})$\\
\\
\textbf{Explanation:} With uniformly random long-range links, the probability of having a link to a region of size $\sqrt{n}$ around the target is $O(1/\sqrt{n})$. Therefore, it takes $\Omega(\sqrt{n})$ steps in expectation to find such a link.
\subsection{Question M3}
Which of the following best describes the small world phenomenon?
\begin{enumerate}[label=(\alph*)]
\item High clustering and high diameter
\item Low clustering and low diameter
\item High clustering and low diameter
\item Low clustering and high diameter
\end{enumerate}
\textbf{Answer:} (c) High clustering and low diameter\\
\\
\textbf{Explanation:} Small world networks combine the local structure of regular networks (high clustering) with the global efficiency of random networks (low diameter/short paths).
\subsection{Question M4}
In Kleinberg's 1D model with $\alpha = 1$, what is the normalizing constant $Z$ approximately equal to?
\begin{enumerate}[label=(\alph*)]
\item $O(1)$
\item $O(\log n)$
\item $O(\sqrt{n})$
\item $O(n)$
\end{enumerate}
\textbf{Answer:} (b) $O(\log n)$\\
\\
\textbf{Explanation:} $Z = \sum_{w \neq v} \frac{1}{d(v,w)} \approx 2\sum_{k=1}^{n/2} \frac{1}{k} \approx 2\ln(n/2) = O(\log n)$. This is the harmonic series approximation.
\section{Short Answer Questions}
\subsection{Question S1}
Consider a Watts-Strogatz model starting with a ring of $n = 1000$ nodes where each node is connected to its $k = 4$ nearest neighbors. If we set the rewiring probability to $p = 0.1$, calculate:
\begin{enumerate}[label=(\alph*)]
\item The initial clustering coefficient $C(0)$
\item The clustering coefficient after rewiring $C(0.1)$
\item The approximate diameter of the network
\end{enumerate}
\textbf{Solution:}\\
\\
(a) Initial clustering coefficient:
$$
C(0) = \frac{3(k-2)}{4(k-1)} = \frac{3(4-2)}{4(4-1)} = \frac{6}{12}=0.5
$$
(b) After rewiring:
$$
C(0.1) = C(0)\cdot (1-p)^3 = 0.5\times (1-0.1)^3 = 0.5\times 0.9^3 = 0.5\times 0.729 = 0.3645
$$
(c) Approximate diameter:
The diameter is approximately $O(\log n)$. Using the formula $\langle d \rangle \approx \frac{\ln N}{\ln \langle k \rangle}$:
$$
\langle d\rangle\approx \frac{\ln 1000}{\ln 4}\approx \frac{6.91}{1.39}\approx 5
$$
The diameter would be roughly $2-3$ times this, so approximately $10-15$.
\subsection{Question S2}
In Kleinberg's 1D model with $n = 10,000$ nodes and $\alpha = 1$, a message is currently at a node that is distance $d = 100$ from the target. What is the probability that the long-range link from this node reaches within distance 50 of the target?\\
\\
\textbf{Solution:}\\
\\
First, calculate the normalizing constant:
$$
Z\approx 2\ln n = 2\ln (10000)\approx 2\times 9.21=18.42
$$
The probability of linking to any specific node at distance $r$ is:
$$
P(r) =\frac{1/r}{Z}
$$
The probability of linking to the interval $[50, 100]$ from the target:
$$
P(\text{reach within $50$}) = \sum_{r=50}^{100}\frac{1/r}{Z}\approx \frac{1}{Z}\int_{50}^{100}\frac{1}{r}\, dr \approx 0.0376
$$
So there's approximately a $3.76\%$ chance of halving the distance in one step.
\subsection{Question S3}
Explain why decentralized search in a 2D grid with Kleinberg's model fails when $\alpha = 0$ (uniform random links). Provide both intuitive reasoning and mathematical justification.\\
\\
\textbf{Solution:}\\
\\
\textbf{Intuitive Reasoning:}\\
With $\alpha = 0$, long-range links are uniformly distributed across all nodes. This means:\\
\\
Links mostly go to very distant nodes (since there are many more distant nodes than nearby ones) These links "overshoot" the target and don't help with navigation
Once near the target, the probability of having a useful shortcut is extremely small\\
\\
\textbf{Mathematical Justification:}\\
Let the current distance to target be $d$. The number of nodes within distance $d/2$ is $O(d^2)$, while the total number of nodes is $O(n)$.
Probability of halving distance: $P = \frac{O(d^2)}{O(n)}$
For $d = \sqrt{n}$: $P = O(1)$ (good progress)
For $d < \sqrt{n}$: $P = o(1)$ (very slow progress)
Therefore, navigation gets "stuck" when $d < \sqrt{n}$, leading to search time $\Omega(\sqrt{n})$.
\subsection{Question S4}
A network follows the degree distribution from Table 3.2 (page 5). If the average degree is $\langle k \rangle = 6$ and the network has $n = 100,000$ nodes:
\begin{enumerate}[label=(\alph*)]
\item Calculate the expected average distance using the approximation formula
\item Explain why real networks might deviate from this approximation
\end{enumerate}
\textbf{Solution:}\\
\\
(a) Using the approximation formula:\\
$$
\langle d\rangle \approx \frac{\ln N}{\ln \langle k\rangle} \approx 6.43
$$
(b) Real networks deviate because:\\
\\
They have degree heterogeneity (hubs), not captured by using average degree
They have community structure, creating longer paths between communities
The approximation assumes tree-like structure, but real networks have cycles
Directedness in networks (like WWW) can increase distances
The formula assumes random connections, but real networks have correlations

\subsection{Question S5}
Prove that in the small world regime of the Watts-Strogatz model (small $p$), it takes very few random edges to create shortcuts. Specifically, show that with $p = \frac{c\log n}{kn}$ for constant $c > 1$, the network is connected with high probability.\\
\\
\textbf{Solution:}\\
\\
Expected number of random edges added:
$$
\mathbb{E}[\text{random edges}]=\frac{1}{2}\cdot kn\cdot p=\frac{2\log n}{2}
$$
This creates a random graph $G(n, \frac{c\log n}{n})$ overlaid on the ring.
From random graph theory, $G(n, \frac{c\log n}{n})$ is connected with high probability when $c > 1$.\\
\\
Key insight: Even though most edges remain local (clustering stays high), the $O(\log n)$ random edges are sufficient to:
\begin{itemize}
    \item Connect different parts of the ring
    \item Create shortcuts that reduce diameter to $O(\log n)$
    \item Maintain the small world property
\end{itemize}


This explains why the transition to small world behavior occurs at very small $p \approx \frac{\log n}{n}$.
\section{Challenging Problems}
\subsection{Question C1}
Consider a variant of Kleinberg's model where each node has $k$ long-range links instead of just one. How does this affect the search time when $\alpha = d$? Provide a rigorous analysis.\\
\\
\textbf{Solution:}\\
With $k$ long-range links and $\alpha = d$:
\textbf{Modified probability:} Each link independently follows $P(u \rightarrow v) \sim d(u,v)^{-d}$
\textbf{Progress probability:} In phase $j$ (distance $\approx 2^j$), probability at least one of $k$ links halves distance:
$$
P(\text{progress}) = 1-(1-\frac{c}{\log n})^k \approx 1-e^{-kc/\log n}\approx \frac{kc}{\log n}
$$
\textbf{Phase length:} Expected steps in phase $j$:
$$
\mathbb{E}[X_j] \approx \frac{\log n}{kc}
$$
\textbf{Total search time:}
$$
\mathbb{E}[Total] = O(\log n)\times \frac{\log n}{k} = O(\frac{(\log n)^2}{k})
$$
Therefore, $k$ long-range links improve search time by a factor of $k$, but the improvement is only linear in $k$.
\subsection{Question C2}
Explain why the principle of deferred decisions is valid in the analysis of Watts-Strogatz navigation. What assumptions does it require?\\
\\
\textbf{Solution:}\\
\\
\textbf{Validity:} The principle states that we can assume random decisions (like edge destinations) are made only when observed, rather than predetermined.\\
\\
\textbf{Why it works:}\\
\\
\textbf{Independence:} Each random edge is chosen independently\\
\\
\textbf{Memorylessness:} The probability distribution doesn't depend on past choices\\
\\
\textbf{Observation order:} The order in which we observe edges doesn't affect their distribution\\
\\
\textbf{Required assumptions:}\\
\begin{itemize}
    \item Edges are rewired independently with probability $p$
    \item Destination of each rewired edge is uniformly random
    \item The search algorithm doesn't use information about unvisited edges
    \item No edge is observed twice (no cycles in search path)
\end{itemize}

\textbf{Application:} This allows us to analyze the search as if each node generates its long-range link only when visited, simplifying probability calculations significantly.
\end{document}