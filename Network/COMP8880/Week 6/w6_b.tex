\documentclass[11pt,a4paper]{article}
\usepackage[margin=1in]{geometry}
\usepackage{amsmath,amsfonts,amssymb}
\usepackage{graphicx}
\usepackage{enumitem}
\usepackage{hyperref}
\usepackage{fancyhdr}
\usepackage{algorithm}
\usepackage{algorithmic}

\pagestyle{fancy}
\fancyhf{}
\rhead{COMP4880/8880 Study Guide}
\lhead{Link Prediction \& Recommender Systems}
\cfoot{\thepage}

\title{\textbf{Link Prediction and Recommender Systems}\\
\large Study Guide for Final Exam}
\author{Based on Course Materials}
\date{Semester 1, 2025}

\begin{document}
\maketitle

\tableofcontents
\newpage

\section{Introduction to Link Prediction}
\textit{Coverage: Document Pages 2-3, 5}

Link prediction is one of the most commercially valuable problems in network science. The fundamental question is: given a snapshot of a network at time $t$, can we predict which new edges will appear at time $t+1$?

\subsection{Problem Formulation}
Consider a network $G = (V, E)$ observed at time $t$. We want to predict missing edges or future connections. This involves:

\begin{itemize}
\item \textbf{Input:} Current network structure $G_t$
\item \textbf{Output:} Ranked list of node pairs likely to form edges
\item \textbf{Applications:} Social network friend suggestions, biological pathway prediction, collaboration networks
\end{itemize}

The challenge lies in distinguishing between truly missing edges and edges that will never exist. We need scoring functions that assign higher scores to more likely connections.

\subsection{Evaluation Methodology}
To evaluate link prediction algorithms, we split the network temporally:
\begin{itemize}
\item \textbf{Training Period:} Use edges up to time $t$ to learn patterns
\item \textbf{Test Period:} Predict edges that appear between $t$ and $t+\Delta t$
\item \textbf{Core Set:} Focus on nodes active in both periods to ensure fair comparison
\end{itemize}

Performance is measured by how well the algorithm ranks true future edges above random pairs.

\section{Link Prediction Methods}
\textit{Coverage: Document Pages 8-10}

Various approaches exist for scoring potential edges. Each method captures different aspects of network structure that indicate likely connections.

\subsection{Distance-Based Methods}

\textbf{Graph Distance:} The intuition is that closer nodes are more likely to connect. We use the negative shortest path length:
$$\text{score}(x,y) = -d(x,y)$$

However, this method has limitations: it assigns the same score to all node pairs at the same distance, ignoring other structural properties.

\subsection{Neighborhood-Based Methods}

These methods assume that nodes with many common neighbors are likely to connect.

\textbf{Common Neighbors:} Simply count shared neighbors:
$$\text{score}(x,y) = |\Gamma(x) \cap \Gamma(y)|$$
where $\Gamma(x)$ denotes the neighbors of node $x$.

\textbf{Jaccard Coefficient:} Normalizes by total neighbors to avoid bias toward high-degree nodes:
$$\text{score}(x,y) = \frac{|\Gamma(x) \cap \Gamma(y)|}{|\Gamma(x) \cup \Gamma(y)|}$$

\textbf{Adamic-Adar Index:} Weights common neighbors by their rarity:
$$\text{score}(x,y) = \sum_{z \in \Gamma(x) \cap \Gamma(y)} \frac{1}{\log|\Gamma(z)|}$$

The intuition is that connections through low-degree nodes provide stronger evidence than connections through hubs.

\textbf{Preferential Attachment:} Based on the "rich get richer" principle:
$$\text{score}(x,y) = |\Gamma(x)| \cdot |\Gamma(y)|$$

\subsection{Path-Based Methods}

\textbf{Katz Index:} Counts paths of all lengths, with exponential decay for longer paths:
$$\text{score}(x,y) = \sum_{\ell=1}^{\infty} \beta^{\ell} \cdot |\text{paths}^{(\ell)}_{x,y}|$$

where $\beta < 1$ is a damping parameter and $\text{paths}^{(\ell)}_{x,y}$ represents paths of length $\ell$ from $x$ to $y$.

\subsection{Random Walk Methods}

\textbf{Hitting Time:} The expected time for a random walk starting at $x$ to reach $y$:
$$H_{x,y} = \mathbb{E}[\text{time to reach } y \text{ starting from } x]$$

The score is $-H_{x,y}$, as shorter hitting times suggest stronger connections.

\textbf{Commute Time:} The expected time for a round trip:
$$C_{x,y} = H_{x,y} + H_{y,x}$$

\textbf{Rooted PageRank:} Modification of PageRank where we periodically jump back to a specific starting node with probability $\alpha$:
$$\pi_x(y) = \alpha \cdot \mathbf{1}_{x}(y) + (1-\alpha) \sum_{z} \pi_x(z) \cdot P_{z,y}$$

where $P_{z,y}$ is the transition probability from $z$ to $y$.

\section{Recommender Systems Foundation}
\textit{Coverage: Document Pages 11-13}

Recommender systems predict user preferences for items they haven't yet encountered. This connects directly to link prediction in bipartite user-item networks.

\subsection{Problem Setup}
We have:
\begin{itemize}
\item \textbf{Users:} Set $U = \{u_1, u_2, \ldots, u_m\}$
\item \textbf{Items:} Set $I = \{i_1, i_2, \ldots, i_n\}$
\item \textbf{Interactions:} Observed user-item interactions (ratings, purchases, clicks)
\end{itemize}

The goal is to recommend items that users will find relevant or rate highly.

\subsection{Data Representations}

\textbf{Activity Sets:} For each user $u$, define $I_u$ as the set of items they've interacted with. Similarly, for each item $i$, define $U_i$ as the set of users who've interacted with it.

\textbf{Interaction Matrix:} Create matrices to capture relationships:
\begin{itemize}
\item \textbf{Rating Matrix } $R$: Entry $R_{u,i}$ contains user $u$'s rating for item $i$
\item \textbf{Consumption Matrix } $C$: Binary matrix where $C_{u,i} = 1$ if user $u$ interacted with item $i$
\end{itemize}

The key challenge is that these matrices are extremely sparse—most users interact with only a tiny fraction of available items.

\subsection{Memory-Based vs Model-Based Approaches}

\textbf{Memory-Based (Neighborhood):} Use similarity functions directly on the original data without learning parameters. These methods find similar users or items and make recommendations based on their preferences.

\textbf{Model-Based:} Learn representations (parameters) for users and items that can generalize beyond the observed data. The original interaction matrix isn't used directly at prediction time.

\section{Similarity Measures for Recommendation}
\textit{Coverage: Document Pages 13-15}

Understanding how to measure similarity between users or items is crucial for memory-based recommender systems.

\subsection{Set-Based Similarity}
When we only have binary interaction data (user liked/didn't like), we can compute similarity using set operations.

\textbf{Jaccard Similarity:} For two items $i$ and $j$:
$$\text{sim}(i,j) = \frac{|U_i \cap U_j|}{|U_i \cup U_j|}$$

This measures the overlap in users who interacted with both items, normalized by the total number of users who interacted with either item.

\subsection{Vector-Based Similarity}
When we have rating data, we can treat user preferences as vectors and compute geometric similarity.

\textbf{Cosine Similarity:} Measures the angle between rating vectors:
$$\text{sim}(u,v) = \frac{R_u \cdot R_v}{||R_u|| \cdot ||R_v||} = \frac{\sum_i R_{u,i} \cdot R_{v,i}}{\sqrt{\sum_i R_{u,i}^2} \cdot \sqrt{\sum_i R_{v,i}^2}}$$

The intuition is that users with similar rating patterns (regardless of scale) are considered similar.

\textbf{Pearson Correlation:} Accounts for different rating scales by centering the data:
$$\text{sim}(u,v) = \frac{\sum_i (R_{u,i} - \bar{R_u})(R_{v,i} - \bar{R_v})}{\sqrt{\sum_i (R_{u,i} - \bar{R_u})^2} \cdot \sqrt{\sum_i (R_{v,i} - \bar{R_v})^2}}$$

where $\bar{R_u}$ is user $u$'s average rating. This handles the case where one user consistently rates higher than another but has similar preferences.

\subsection{The Netflix Prize Context}
The Netflix Prize (2006-2009) was a landmark competition that advanced recommender systems research significantly. Netflix released 100 million movie ratings and challenged teams to improve their recommendation accuracy by 10\%. The competition highlighted the importance of matrix factorization techniques and ensemble methods.

\section{Matrix Factorization for Recommendations}
\textit{Coverage: Document Pages 17-20}

Matrix factorization became the dominant approach for recommender systems after the Netflix Prize, offering a powerful way to handle sparse data and discover latent factors.

\subsection{Basic Matrix Factorization}
The core idea is to approximate the sparse rating matrix $R$ (size $|U| \times |I|$) as the product of two dense matrices:
$$R \approx \gamma_U \cdot \gamma_I^T$$

where:
\begin{itemize}
\item $\gamma_U$ is $|U| \times K$ (user factor matrix)
\item $\gamma_I$ is $|I| \times K$ (item factor matrix)  
\item $K$ is much smaller than $|U|$ and $|I|$ (typically 50-200)
\end{itemize}

Each user and item is represented by a $K$-dimensional latent factor vector. The predicted rating is:
$$\hat{R}_{u,i} = \gamma_u \cdot \gamma_i = \sum_{k=1}^K \gamma_{u,k} \cdot \gamma_{i,k}$$

\subsection{Relationship to SVD}
This approach is inspired by Singular Value Decomposition (SVD), which decomposes a matrix as $M = U\Sigma V^T$. However, standard SVD has two major limitations for recommendation:

\textbf{Missing Data Problem:} SVD requires complete matrices, but rating matrices are extremely sparse (typically $<1\%$ filled).

\textbf{Computational Complexity:} Computing SVD for matrices with millions of users and items is computationally prohibitive.

Matrix factorization for recommendation addresses these by only fitting the observed entries and using iterative optimization.

\subsection{Learning the Factorization}
We minimize the squared error on observed ratings plus regularization:
$$\min_{\gamma_U, \gamma_I} \sum_{(u,i) \in \text{observed}} (R_{u,i} - \gamma_u \cdot \gamma_i)^2 + \lambda(||\gamma_u||^2 + ||\gamma_i||^2)$$

The regularization term $\lambda$ prevents overfitting by keeping the factor vectors small.

This can be solved using:
\begin{itemize}
\item \textbf{Gradient Descent:} Iteratively update factors in the direction that reduces error
\item \textbf{Alternating Least Squares:} Fix user factors and solve for item factors, then vice versa
\end{itemize}

\subsection{Interpretation of Latent Factors}
The $K$ dimensions can often be interpreted as underlying preferences or item characteristics:
\begin{itemize}
\item For movies: genre preferences (comedy vs drama), recency (old vs new), target audience (adult vs family)
\item For products: price sensitivity, brand preference, functionality needs
\end{itemize}

Users and items with similar factor vectors will have high predicted ratings, capturing collaborative filtering effects automatically.

\section{Advanced Topics and Connections}
\textit{Coverage: Document Pages 19-21}

\subsection{Link Prediction as Matrix Factorization}
The connection between link prediction and collaborative filtering runs deep. Both problems can be viewed as predicting missing entries in sparse matrices:

\begin{itemize}
\item \textbf{Link Prediction:} Predict missing entries in the adjacency matrix
\item \textbf{Collaborative Filtering:} Predict missing entries in the user-item rating matrix
\end{itemize}

This unified view, termed "dyadic prediction," allows methods from one domain to be applied to the other.

\subsection{Method Comparison}
Different link prediction methods excel in different scenarios:

\begin{itemize}
\item \textbf{Neighborhood methods} work well for networks with strong clustering
\item \textbf{Path-based methods} capture longer-range dependencies
\item \textbf{Matrix factorization} handles large, sparse networks effectively
\item \textbf{Random walk methods} incorporate global network structure
\end{itemize}

The choice depends on network characteristics, computational constraints, and available side information.

\subsection{Practical Considerations}
Real-world systems face additional challenges:

\textbf{Cold Start:} How to make recommendations for new users or items with no interaction history.

\textbf{Scalability:} Methods must handle millions of users and items efficiently.

\textbf{Temporal Dynamics:} User preferences and item popularity change over time.

\textbf{Evaluation Metrics:} Beyond accuracy, we care about diversity, novelty, and fairness of recommendations.

\section{Key Takeaways for the Exam}

\subsection{Core Concepts to Remember}
\begin{enumerate}
\item Link prediction methods exploit different network structures—understand when each is appropriate
\item Similarity measures in recommender systems capture different aspects of user/item relationships
\item Matrix factorization discovers latent factors that explain observed interactions
\item Both problems fundamentally involve predicting missing edges in bipartite graphs
\item Evaluation requires careful temporal splitting to avoid data leakage
\end{enumerate}

\subsection{Mathematical Formulations}
Be prepared to:
\begin{itemize}
\item Calculate common neighbors, Jaccard coefficient, and Adamic-Adar scores
\item Explain the matrix factorization objective function
\item Compute cosine and Pearson similarities
\item Understand the relationship between SVD and collaborative filtering
\end{itemize}

\subsection{Practical Applications}
These techniques power real-world systems including social media friend suggestions, e-commerce recommendations, content discovery platforms, and scientific collaboration prediction.

\end{document}