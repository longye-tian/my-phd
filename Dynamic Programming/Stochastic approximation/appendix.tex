\section{Appendix}
\subsection{Direct comparison of the simplified setup and \cite{tsitsiklis1994asynchronous}}\label{app:comparison}
\subsubsection{Model Setup in \cite{tsitsiklis1994asynchronous}}

We consider iterative updates of a vector $x \in \mathbb{R}^n$ to solve the fixed-point equation $F(x) = x$, where $F: \mathbb{R}^n \mapsto \mathbb{R}^n$ with component mappings $F_i: \mathbb{R}^n \mapsto \mathbb{R}$.\\
\\
Let $x(t)$ denote the state at discrete time $t \in \mathbb{N}$, with components $x_i(t)$. For each component $i$, we have:
\begin{equation}\label{eq:update_rule_tsi}
x_i(t + 1) = 
\begin{cases}
x_i(t), & t \notin T^i \\
x_i(t) + \alpha_i(t)(F_i(x^i(t)) - x_i(t) + w_i(t)), & t \in T^i
\end{cases}
\end{equation}
where:
\begin{itemize}
\item $T^i \subset \mathbb{N}$ is the set of update times for component $i$
\item $\alpha_i(t) \in [0,1]$ is the stepsize parameter
\item $w_i(t)$ is a noise term
\item $x^i(t) = (x_1(\tau_1^i(t)), \ldots, x_n(\tau_n^i(t)))$ contains possibly outdated information with $0 \leq \tau_j^i(t) \leq t$
\end{itemize}
All variables are defined on a probability space $(\Omega, \mathcal{F}, P)$ with an increasing sequence of $\sigma$-fields $\{\mathcal{F}(t)\}_{t=0}^{\infty}$ representing the algorithm's history.\\
\\
For any positive vector $v = (v_1, \ldots, v_n)$, we define the weighted maximum norm:
\begin{equation*}
\|x\|_v = \max_i \frac{|x_i|}{v_i}, \quad x \in \mathbb{R}^n
\end{equation*}
When $v = (1,\ldots,1)$, this is the standard maximum norm $\|\cdot\|_{\infty}$.
\subsubsection{Simplified model setup}
Let $x(t)$ denote the state at discrete time $t\in\mathbb{N}$ with component $x_i(t)$. For each component, we have
\begin{equation}
    x_i(t+1) = (1-\alpha_i(t))x_i(t) + \alpha_i(t)(F_i(\textcolor{blue}{x(t)}) + w_i(t)) 
\end{equation}\label{eq:x_t}

where
\begin{itemize}
    \item $\alpha_i(t) \in [0,1]$ is the stepsize parameter
    \item $w_i(t)$ is a noise term
\end{itemize}
\begin{remark}
    In this simplified setup, we can write \autoref{eq:x_t} into vector form, i.e.,
    $$
    \mathbf{x}(t+1) =(I-\mathbf{A}(t)) \mathbf{x}(t) + \mathbf{A}(t)(\mathbf{F}(\mathbf{x}(t)) +\mathbf{w}(t))
    $$
    where
    $$
    \mathbf{x}(t) = \begin{pmatrix}
        x_1(t)\\
        \vdots\\
        x_n(t)
    \end{pmatrix},\mathbf{A}(t) = \begin{pmatrix}
        \alpha_1(t) & 0& \cdots & 0\\
        0& \alpha_2(t) & \cdots & 0\\
        \vdots & \vdots & \ddots & 0\\
        0 &0 &\cdots & \alpha_n(t)
    \end{pmatrix},\mathbf{F}(\mathbf{x}(t)) = \begin{pmatrix}
        F_1(\mathbf{x}(t))\\
        \vdots\\
        F_n(\mathbf{x}(t))
    \end{pmatrix},\mathbf{w}(t) = \begin{pmatrix}
        w_1(t)\\
        \vdots\\
        w_n(t)
    \end{pmatrix}
    $$
    But for the sake of doing one thing at a time, we keep the notation in \cite{tsitsiklis1994asynchronous} for now.
\end{remark}
All variables are defined on a probability space $(\Omega, \mathcal{F}, P)$ with an increasing sequence of $\sigma$-fields $\{\mathcal{F}(t)\}_{t=0}^{\infty}$ representing the algorithm's history. \\
\\
For any positive vector $v = (v_1, \ldots, v_n)$, we define the weighted maximum norm:
\begin{equation*}
\|x\|_v = \max_i \frac{|x_i|}{v_i}, \quad x \in \mathbb{R}^n
\end{equation*}
\noindent When $v = (1,\ldots,1)$, this is the standard maximum norm $\|\cdot\|_{\infty}$.\\
\\
\noindent Compared to \autoref{eq:update_rule_tsi}, no information is outdated, hence, we have $x^i(t) = x(t)$. And we don't consider the update time for different cases. As mentioned in \cite{tsitsiklis1994asynchronous}, this is a special case, hene all the theorems work fine under this setup.

\subsubsection{Assumptions in \cite{tsitsiklis1994asynchronous}}

\noindent
\textcolor{blue}{\textbf{Assumption 1} (Total Asynchronism).
For any $i$ and $j$, $\lim_{t \to \infty} \tau_j^i(t) = \infty$, with probability 1.}

\vspace{1em}

\noindent
\textbf{Assumption 2} (Statistical Properties).
\begin{enumerate}
\item[(a)] $x(0)$ is $\mathcal{F}(0)$-measurable;
\item[(b)] For every $i$ and $t$, $w_i(t)$ is $\mathcal{F}(t+1)$-measurable;
\item[(c)] For every $i$, $j$, and $t$, $\alpha_i(t)$ and \textcolor{blue}{$\tau_j^i(t)$} are $\mathcal{F}(t)$-measurable;
\item[(d)] For every $i$ and $t$, we have $\mathbb{E}[w_i(t) \mid \mathcal{F}(t)] = 0$;
\item[(e)] There exist constants $A$ and $B$ such that
$\mathbb{E}[w_i^2(t) \mid \mathcal{F}(t)] \leq A + B \max_j \max_{\tau \leq t} |x_j(\tau)|^2$, $\forall i, t$.
\end{enumerate}

\vspace{1em}

\noindent
\textbf{Assumption 3} (Stepsize Conditions).
\begin{enumerate}
\item[(a)] For every $i$, $\sum_{t=0}^{\infty} \alpha_i(t) = \infty$, w.p.1;
\item[(b)] There exists a constant $C$ such that for every $i$, $\sum_{t=0}^{\infty} \alpha_i^2(t) \leq C$, w.p.1.
\end{enumerate}

\vspace{1em}

\noindent
\textbf{Assumption 5} (Contraction).
There exists a vector $x^* \in \mathbb{R}^n$, a positive vector $v$, and a scalar $\beta \in [0,1)$, such that
\begin{equation}
\|F(x) - x^*\|_v \leq \beta \|x - x^*\|_v, \quad \forall x \in \mathbb{R}^n.
\end{equation}

\vspace{1em}

\noindent
\textbf{Assumption 6} (Boundedness).
There exists a positive vector $v$, a scalar $\beta \in [0,1)$, and a scalar $D$ such that
\begin{equation}
\|F(x)\|_v \leq \beta\|x\|_v + D, \quad \forall x \in \mathbb{R}^n.
\end{equation}
\begin{remark}
    We don't present assumption 4 as it is not required for the theorem of interest. Later, in the simplified version, we will re-enumerate the number. Here is the enumeration to match \cite{tsitsiklis1994asynchronous}
\end{remark}
\subsubsection{Simplified assumptions}
\noindent
\textbf{Assumption 1} (Simplified).
This assumption is no longer needed since no information is outdated.
\vspace{1em}

\noindent
\textbf{Assumption 2} (Statistical Properties).
\begin{enumerate}
\item[(a)] $x(0)$ is $\mathcal{F}(0)$-measurable;
\item[(b)] For every $i$ and $t$, $w_i(t)$ is $\mathcal{F}(t+1)$-measurable;
\item[(c)] For every $i$ and $t$, $\alpha_i(t)$ is $\mathcal{F}(t)$-measurable;
\item[(d)] For every $i$ and $t$, we have $\mathbb{E}[w_i(t) \mid \mathcal{F}(t)] = 0$;
\item[(e)] There exist constants $A$ and $B$ such that
$\mathbb{E}[w_i^2(t) \mid \mathcal{F}(t)] \leq A + B \max_j \max_{\tau \leq t} |x_j(\tau)|^2$, $\forall i, t$.
\end{enumerate}

\vspace{1em}

\noindent
\textbf{Assumption 3} (Stepsize Conditions).
\begin{enumerate}
\item[(a)] For every $i$, $\sum_{t=0}^{\infty} \alpha_i(t) = \infty$, w.p.1;
\item[(b)] There exists a constant $C$ such that for every $i$, $\sum_{t=0}^{\infty} \alpha_i^2(t) \leq C$, w.p.1.
\end{enumerate}

\vspace{1em}

\noindent
\textbf{Assumption 5} (Contraction).
There exists a vector $x^* \in \mathbb{R}^n$, a positive vector $v$, and a scalar $\beta \in [0,1)$, such that
\begin{equation}
\|F(x) - x^*\|_v \leq \beta \|x - x^*\|_v, \quad \forall x \in \mathbb{R}^n.
\end{equation}

\vspace{1em}

\noindent
\textbf{Assumption 6} (Boundedness).
There exists a positive vector $v$, a scalar $\beta \in [0,1)$, and a scalar $D$ such that
\begin{equation}
\|F(x)\|_v \leq \beta\|x\|_v + D, \quad \forall x \in \mathbb{R}^n.
\end{equation}