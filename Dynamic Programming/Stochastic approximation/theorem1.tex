\section{Full proof for theorem 1}

We need three lemmas as in \cite{jaakkola1993convergence}.

\begin{lemma}[Lemma 1 of \cite{jaakkola1993convergence}]
A random process 
$$
w_{n+1}(x) = (1-\alpha_n(x))w_n(x)+\beta_n(x)r_n(x)
$$
converges to zero with probability one if the following conditions are satisfied:
\begin{enumerate}
    \item $\sum_{n=1}^\infty \alpha_n(x) = \infty, \sum_{n=1}^\infty \alpha_n^2(x)<\infty$ and $\sum_{n=1}^\infty \beta_n(x) = \infty$, and $\sum_{n=1}^\infty \beta_n^2(x)<\infty$ uniformly over $x$ with probability one.
    \item $\mathbb{E}(r_n(x)|P_n,\beta_n) = 0$ and $\mathbb{E}(r_n^2(x)|P_n,\beta_n)\le C$ with probability one where
\end{enumerate}
    
\end{lemma}

\section{Extension}
\subsection{Related Definitions}
\begin{definition}[Eventual contraction]
We call a self-map $S$ on a subset $V$ of a Banach lattice $E$ if there exists a positive linear operator $K:E \to E$ such that
\begin{itemize}
    \item $\rho(K)<1$ 
    \item $|Sv-Sw|\le K|v-w|$  for all $v,w\in V$
\end{itemize}

\end{definition}
\begin{lemma}[Gelfand's formula]\label{lm:gelfand}
    If $B$ is any square matrix and $\|\cdot\|$ is any matrix norm, then
    $$
    \rho(B)^k\le \|B^k\|\quad \text{for all $k\in\mathbb{N}$}
    $$
    $$
    \|B^k\|^{1/k}\to \rho(B)\text{ as $k\to\infty$}
    $$
\end{lemma}
\begin{corollary}\label{col:gelfand col}
    If $B$ is any square matrix and $\|\cdot\|$ is any matrix form, then if there exists $n\in\mathbb{N}$ such that 
    $$
    \|B^n\|<1
    $$
    this implies $\rho(B)<1$.
\end{corollary}
\begin{proof}
    Using \autoref{lm:gelfand}, we have
    $$
    \rho(B)^n \le \|B^n\|<1
    $$
    Hence, $\rho(B)< \sqrt[n]{1}=1$.
\end{proof}
\begin{lemma}
Let $A$ be a $n$-dimensional nonnegative square matrix with spectral radius $\rho(A)<1$. Then there exists a strictly positive matrix $B$ such that 
$$
A< B\text{ and $\rho(B)<1$}
$$
\end{lemma}
\begin{proof}
    Let $J$ denote the $n$-dimensional square matrix with every entry equals to 1. We construct $B = A +\epsilon  J$. We show that there exists $1>\epsilon>0$ such that $\rho(B)<1$.\\
    \\
    Using the Gelfand's formula, we have there exists $N\in\mathbb{N}$ such that for all $n\ge N$, $\|A^n\|< 1$. Fix $n\ge N$. We set $\delta:= 1-\|A^n\|$.\\
    \\
    Moreover, we have 
    \begin{align*}
        \|B^n\| &= \|(A+\epsilon J)^n\|\\
        &= \|A^n +  \epsilon (\Gamma_{1,1} +\cdots + \Gamma_{1,C^n_1})+\cdots +  \epsilon^{n-1}(\Gamma_{n-1, 1} + \cdots \Gamma_{n-1, C^{n}_{n-1}}) +\epsilon^n J^n\|
    \end{align*}
    for some square matrix $\Gamma_{i,j}$. 
    \begin{remark}
    To motive this step, we have for $n=2$,
    \begin{align*}
        (A + \epsilon J)^2  &= A^2 + \epsilon AJ + \epsilon J A + \epsilon^2 J^2\\
        &= A^2 + \epsilon (AJ +JA) +\epsilon ^2 J^2
    \end{align*}
    Hence, we have $\Gamma_{1,1} = AJ$ and $\Gamma_{1,2} = JA$
\end{remark}
    \noindent Then by triangle inequality, we have
    \begin{align*}
         \|B^n\| &\le \|A^n\| + \sum_{k=1}^{n-1} \epsilon^k \left(\sum_{j=1}^{C^n_k} \|\Gamma_{k,j}\|\right) + \epsilon^n \|J^n\|
    \end{align*}
    Let 
    $$
    M:=\max_{1\le k,j\le n}\{\|\Gamma_{k,j}\|, \|J^n\|\}
    $$
    $$
    \gamma: =\max_{1\le k\le n} C^n_k
    $$
    By finite dimension, we have $M$ is well-defined and finite. This gives
   \begin{align*}
         \|B^n\| &\le \|A^n\| + \gamma M\sum_{k=1}^{n} \epsilon^k\\
         &<\|A^n\| + \gamma M n\epsilon \tag{$0<\epsilon<1$}
    \end{align*}
    Let $\epsilon <\frac{\delta}{\gamma Mn}$. Then, we have
    $$
    \|B^k\| = \|(A+\epsilon J)^k\|< 1
    $$
    By \autoref{col:gelfand col}, this implies $\rho(B)<1$.
\end{proof}

In the finite dimensional case, we can extend \textbf{Assumption 3} (contraction) to \textbf{Assumption $3'$} (eventually contraction). Here is the comparison: \\
\\
\textbf{Assumption $3$} (Contraction).
There exists a vector $x^* \in \mathbb{R}^n$, a positive vector $v$, and a scalar $\beta \in [0,1)$, such that
\begin{equation}
\|F(x) - x^*\|_v \leq \beta \|x - x^*\|_v, \quad \forall x \in \mathbb{R}^n.
\end{equation} holds.\\
\\
\textbf{Assumption $3'$}  (Eventual Contraction).
There exists a vector $x^* \in \mathbb{R}^n$, and positive linear operator $K$ such that
\begin{equation}
|F(x) - x^*| \leq K |x - x^*|, \quad \forall x \in \mathbb{R}^n.
\end{equation} holds.\\
\\
