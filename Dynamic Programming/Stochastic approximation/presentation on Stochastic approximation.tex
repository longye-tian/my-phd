%%%%%%%%%%%%%%%%%%%%%%%%%%%%%%%%%%%%%%%%%
% Professional Mathematical Presentation Template
% 
% This template uses the beamer class with the Madrid theme
% and a custom color scheme for a clean, professional look
% that works well with mathematical content.
%%%%%%%%%%%%%%%%%%%%%%%%%%%%%%%%%%%%%%%%%
\documentclass[aspectratio=169]{beamer} % 16:9 aspect ratio (modern)
% Theme settings
\usetheme{Madrid}
\usecolortheme{default}
\usepackage[dvipsnames]{xcolor}
\definecolor{primcolor}{RGB}{25,50,100} % Dark blue
\setbeamercolor{structure}{fg=primcolor}
\setbeamercolor{frametitle}{bg=primcolor!15, fg=primcolor}
\setbeamercolor{title}{fg=white} % White title text for contrast
\setbeamercolor{subtitle}{fg=white} % White subtitle text
\setbeamercolor{author}{fg=primcolor} % White author text
\setbeamercolor{date}{fg=primcolor} % White date text
\setbeamercolor{institute}{fg=primcolor} % White institute text
% Font settings
\usefonttheme{professionalfonts}
\usefonttheme{serif}
% Package imports
\usepackage{amsmath, amsfonts, amssymb, amsthm} % Math packages
\usepackage{mathtools} % Enhanced math tools
\usepackage{bm} % Bold math symbols
\usepackage{graphicx} % For images
\usepackage{booktabs} % Professional tables
\usepackage{tikz} % For diagrams
\usetikzlibrary{arrows, positioning, matrix, decorations.pathreplacing}
% Use beamer's theorem styles
\setbeamertemplate{theorem}[ams style]
\setbeamertemplate{theorems}[numbered]
% Remove navigation symbols
\setbeamertemplate{navigation symbols}{}
% Custom footer
\setbeamertemplate{footline}{
  \leavevmode%
  \hbox{%
  \begin{beamercolorbox}[wd=.333333\paperwidth,ht=2.25ex,dp=1ex,center]{author in head/foot}%
    \usebeamerfont{author in head/foot}\insertshortauthor
  \end{beamercolorbox}%
  \begin{beamercolorbox}[wd=.333333\paperwidth,ht=2.25ex,dp=1ex,center]{title in head/foot}%
    \usebeamerfont{title in head/foot}\insertshorttitle
  \end{beamercolorbox}%
  \begin{beamercolorbox}[wd=.333333\paperwidth,ht=2.25ex,dp=1ex,right]{date in head/foot}%
    \usebeamerfont{date in head/foot}\insertshortdate{}\hspace{2em}
    \insertframenumber{} / \inserttotalframenumber\hspace{2ex} 
  \end{beamercolorbox}}%
  \vskip0pt%
}
% Title information
\title[SA]{Note on Stochastic Approximation}
\subtitle{Extending Tsitsiklis 1994}
\author[Longye]{Longye Tian \\ \texttt{longye.tian@anu.edu.au}}
\institute[ANU]{Australian National University\\ School of Economics}
\date{May 16th, 2025}
\DeclareFontFamily{U}{mathx}{\hyphenchar\font45}
\DeclareFontShape{U}{mathx}{m}{n}{
      <5> <6> <7> <8> <9> <10>
      <10.95> <12> <14.4> <17.28> <20.74> <24.88>
      mathx10
      }{}
\DeclareSymbolFont{mathx}{U}{mathx}{m}{n}
\DeclareMathSymbol{\bigtimes}{1}{mathx}{"91}

\begin{document}

% Title frame
\begin{frame}
  \titlepage
\end{frame}
\begin{frame}{Big Picture}
\begin{itemize}
    \item We want to estimate some unknown function $x^*$
    \item We have an estimate $x(t)$ at time $t$
    \item At each time $t$, we have noisy observation $F(x(t))+w(t)$
    \begin{itemize}
        \item $F(x(t))$: we can think this as the fixed point equation
        $$
        F(x^*) = x^*
        $$
        \item $w(t)$: a random noise comes with the observation
    \end{itemize}
    \item Stochastic approximation algorithm ($x(t)\to x^*$ as $t\to\infty$)
    $$
    x(t+1) = (1-\alpha(t)) x(t) + \alpha(t) (F(x(t))+w(t))
    $$
    or
    $$
    x(t+1) = x(t) + \alpha(t) \left[F(x(t)) + w(t) -x(t)\right]
    $$
\end{itemize}
\end{frame}
\begin{frame}{Motivating example: Q-learning}
    \begin{itemize}
        \item We want to estimate the unknown $Q^*(s,a)$
        \begin{itemize}
            \item $Q^*(s,a)$ is the maximal expected lifetime rewards given state $s$ and action $a$.
            \item with \textbf{known} reward function $r(s,a)$ and transition probability, we can use DP method to compute $Q^*$
            $$
            Q^*(s,a) = \textcolor{blue}{r(s,a)} + \beta \sum_{s'} \max_{a'}Q^*(s',a') \textcolor{blue}{P(s'|s,a)}
            $$
            \item Sometime, we only observe
            \begin{itemize}
                \item one realization of the random variable for reward $R$
                \item one next state $s'$ not all possible next states
                \item use current estimate of $Q$ not $Q^*$
            \end{itemize}
        \end{itemize}
        \item at each time $t$, we observe 
        $$
        R(t) + \beta \max_{a'} Q(s',a')
        $$
        where $F(Q(s,a))=\mathbb{E}(R+\beta \max_{a'}Q(s',a')|s,a)$
    \end{itemize}
\end{frame}
\begin{frame}{Motivating example: Q-learning}
The stochastic approximation algorithm in Q-learning
$$
Q(s,a) \leftarrow Q(s,a) + \alpha(t) \left[R + \beta \max_{a'} Q(s',a')- Q(s,a)\right]
$$
    
\end{frame}
% Outline frame
\begin{frame}{Outline}
\begin{itemize}
    \item Simplified setup compared to Tsitsiklis 1994
    \item Lemma 1 and Robbins-Siegmund Theorem
    \item Theorem 1 in Tsitsiklis 1994
    \item Theorem 3 in Tsitsiklis 1994
    \item Extension to Eventual contraction assumption
\end{itemize}
\end{frame}
\begin{frame}{Simplified Setup}
Let $x(t)$ denote the state at discrete time $t\in\mathbb{N}$ with component $x_i(t)$. For each component, we have
$$
x_i(t+1) = (1-\alpha_i(t))x_i(t) + \alpha_i(t)(F_i(x(t)) + w_i(t)) 
$$
where
\begin{itemize}
    \item $\alpha_i(t) \in [0,1]$ is the stepsize parameter
    \item $w_i(t)$ is a noise term
\end{itemize}
All variables are defined on a probability space $(\Omega, \mathcal{F}, P)$ with an increasing sequence of $\sigma$-fields $\{\mathcal{F}(t)\}_{t=0}^{\infty}$ representing the algorithm's history. \\
\\
For any positive vector $v = (v_1, \ldots, v_n)$, we define the weighted maximum norm:
\begin{equation}\label{eq:weighted norm}
\|x\|_v = \max_i \frac{|x_i|}{v_i}, \quad x \in \mathbb{R}^n
\end{equation}
    
\end{frame}

\begin{frame}{Simplified Setup - Assumption 1 - need for all theorems}
We assume
    \begin{enumerate}
\item[(a)] $x(0)$ is $\mathcal{F}(0)$-measurable;
\item[(b)] For every $i$ and $t$, $w_i(t)$ is $\mathcal{F}(t+1)$-measurable;
\item[(c)] For every $i$ and $t$, $\alpha_i(t)$ is $\mathcal{F}(t)$-measurable;
\item[(d)] For every $i$ and $t$, we have $\mathbb{E}[w_i(t) \mid \mathcal{F}(t)] = 0$;
\item[(e)] There exist constants $A$ and $B$ such that
$\mathbb{E}[w_i^2(t) \mid \mathcal{F}(t)] \leq A + B \max_j \max_{\tau \leq t} |x_j(\tau)|^2$, $\forall i, t$.
\end{enumerate}
    
\end{frame}
\begin{frame}{Assumption 2 - need for all theorems}
We assume
\begin{enumerate}
\item[(a)] For every $i$, $\sum_{t=0}^{\infty} \alpha_i(t) = \infty$, w.p.1;
\item[(b)] There exists a constant $C$ such that for every $i$, $\sum_{t=0}^{\infty} \alpha_i^2(t) \leq C$, w.p.1.
\end{enumerate}
\end{frame}
\begin{frame}{Assumption 3 - contraction}
    There exists a vector $x^* \in \mathbb{R}^n$, a positive vector $v$, and a scalar $\beta \in [0,1)$, such that
\begin{equation}
\|F(x) - x^*\|_v \leq \beta \|x - x^*\|_v, \quad \forall x \in \mathbb{R}^n.
\end{equation}
\end{frame}
\begin{frame}{Assumption 4 - boundedness}
    There exists a positive vector $v$, a scalar $\beta \in [0,1)$, and a scalar $D$ such that
\begin{equation}
\|F(x)\|_v \leq \beta\|x\|_v + D, \quad \forall x \in \mathbb{R}^n.
\end{equation}
\end{frame}
\begin{frame}{Remark: Assumption 3 implies Assumption 4}
Notice that Assumption 3 implies Assumption 4:
        \begin{align*}
        \|F(x)\|_v &\le \|F(x)-x^*\|_v + \|x^*\|_v\tag{$\Delta$ ineq.}\\
        &\le \beta \|x-x^*\|_v+\|x^*\|_v\tag{Assumption 3}\\
        &\le \beta\|x\|_v + (1+\beta) \|x^*\|_v\tag{$\Delta$ ineq.}
    \end{align*}
    Let $D:= (1+\beta)\|x^*\|_v$
    
\end{frame}

\begin{frame}{Robbins-Siegmund Theorem (Almost supermartingale)}
\begin{theorem}[Robbins-Siegmund]\label{thm:robbins siegmund}
Let $(\Omega, \mathcal{F}, \mathbb{P})$ be a probability space and $\{\mathcal{F}_n\}_{n=0}^{\infty}$ be a filtration. Let $\{V_n,\beta_n, \xi_n, \zeta_n\}_{n=0}^\infty$ be sequences of non-negative random variables adapted to $\{\mathcal{F}_n\}_{n=0}^{\infty}$ such that:
$$
\mathbb{E}[V_{n+1} \mid \mathcal{F}_n] \leq (1+\beta_n)V_n+\xi_n-\zeta_n\quad \text{a.s. for all } n \geq 0
$$
where
\begin{itemize}
    \item $\sum_{n=0}^\infty \beta_n<\infty$ almost surely
    \item $\sum_{n=0}^\infty \xi_n<\infty$ almost surely
\end{itemize}
Then:
\begin{itemize}
    \item $\lim_{n\to\infty}V_n = V_\infty$ exists and is finite almost surely
    \item $\sum_{n=0}^\infty \zeta_n<\infty$ almost surely
\end{itemize}
\end{theorem}
    
\end{frame}

\begin{frame}{Lemma 1}
\begin{lemma}\label{lm:1}
Let $\{\mathcal{F}(t)\}$ be an increasing sequence of $\sigma$-fields. For each $t$, let $\alpha(t)$, $w(t-1)$, and $B(t)$ be $\mathcal{F}(t)$-measurable scalar random variables. Let $C$ be a deterministic constant. Suppose that the following hold with probability 1:
\begin{enumerate}
\item[(a)] $\mathbb{E}[w(t) \mid \mathcal{F}(t)] = 0$;
\item[(b)] $\mathbb{E}[w^2(t) \mid \mathcal{F}(t)] \leq B(t)$;
\item[(c)] $\alpha(t) \in [0,1]$;
\item[(d)] $\sum_{t=0}^{\infty} \alpha(t) = \infty$;
\item[(e)] $\sum_{t=0}^{\infty} \alpha^2(t) \leq C$.
\end{enumerate}
Suppose that the sequence $\{B(t)\}$ is bounded with probability 1. Let $W(t)$ satisfy the recursion
\begin{equation}
W(t+1) = (1 - \alpha(t))W(t) + \alpha(t)w(t).
\end{equation}
Then $\lim_{t \to \infty} W(t) = 0$, with probability 1.
\end{lemma}
    
\end{frame}

\begin{frame}{Proof Sketch for Lemma 1}
    The proof is based on Robbins-Siegmund Theorem
\begin{enumerate}
    \item We use the squared process $V(t) =W^2(t)$ and show that the squared process fits the condition of Robbins-Siegmund Theorem
\begin{align*}
\mathbb{E}[V(t+1) \mid \mathcal{F}(t)] &\leq V(t) + \alpha^2(t) K -\alpha(t) V(t)\\
\mathbb{E}[V_{n+1} \mid \mathcal{F}_n] &\leq (1+\beta_n)V_n+\xi_n-\zeta_n\quad \text{a.s. for all } n \geq 0
\end{align*}
    \item Use Robbins-Siegmund Theorem to get convergence $V(t) \to V_\infty$ and $\sum_{t=0}^\infty \zeta_t = \sum_{t=0}^\infty \alpha (t)V(t)<\infty$ almost surely.
    \item Prove $V_\infty=0$ almost surely by contradiction, hence the original process converges to zero almost surely.
    \begin{align*}
         P\{V_\infty\ge2\epsilon\}>\delta &\implies P(V(t)\ge \epsilon, t\ge T)>\delta\\
         &\implies P\left(\sum_{t=0}^\infty \alpha (t)V(t)=\infty\right)>\delta
    \end{align*}
\end{enumerate}
\end{frame}

\begin{frame}{Main Theorem 1 in Tsitsiklis 1994}
Let $(\Omega, \mathcal{F},P)$ be a probability space with filtration $\{\mathcal{F}_t\}_{t=0}^\infty$. Let $x(t)$ denote the state at discrete time $t\in\mathbb{N}$ with component $x_i(t)$. For each component, we have
$$
x_i(t+1) = (1-\alpha_i(t))x_i(t) + \alpha_i(t)(F_i(x(t)) + w_i(t)) 
$$
If Assumption 1,2,4 holds, then, the sequence $x(t)$ is bounded with probability 1.

\end{frame}
\begin{frame}{Proof Sketch}
\begin{enumerate}
    \item Create a growing envelope $G(t)$ to track the growth of $x(t)$
    \item Use this tracking and growing envelope to normalize the noise and this normalized noise fits the condition of lemma 1.
    \item We use lemma 1 to show that the normalized noise converges to 0
    \item Setup the contradiction by selecting a time $t_0$ that the noise is very small for all $t\ge t_0$
    \item Derive the contradiction by showing the growing envelope is stablized after $t_0$ by induction
\end{enumerate}
    
\end{frame}

\begin{frame}{Main Theorem 2 in Tsitsiklis 1994}
Let $(\Omega, \mathcal{F},P)$ be a probability space with filtration $\{\mathcal{F}_t\}_{t=0}^\infty$. Let $x(t)$ denote the state at discrete time $t\in\mathbb{N}$ with component $x_i(t)$. For each component, we have
$$
x_i(t+1) = (1-\alpha_i(t))x_i(t) + \alpha_i(t)(F_i(x(t)) + w_i(t)) 
$$
If Assumption 1,2,3 holds,then, the sequence $x(t)$ converges to $x^*$ with probability 1.
    
\end{frame}
\begin{frame}{Proof Sketch}
\begin{enumerate}
    \item Show that $x(t)$ is bounded using Main theorem 1
    \item Create a sequence of decreasing bounds $D_0, D_1, D_2,\cdots$ that converges to zero
    \item Prove using induction that for each $k$, the proess eventually stays within the bounds given by $D_k$, this is the outer induction.
    \item To prove the induction step in the outer induction, we use an inner induction to show that the process eventually moves to $D_{k+1}$.
\end{enumerate}
    
\end{frame}

\begin{frame}{Extension to Eventual Contraction}
\textbf{Assumption 3 - Contraction:}\\
There exists a vector $x^* \in \mathbb{R}^n$, a positive vector $v$, and a scalar $\beta \in [0,1)$, such that\\
\begin{equation}
\|F(x) - x^*\|_v \leq \beta \|x - x^*\|_v, \quad \forall x \in \mathbb{R}^n.
\end{equation}
\\
\textbf{Assumption 3+ - Eventual contraction:}\\
There exists a vector $x^* \in \mathbb{R}^n$, and positive linear operator $K$ with spectral radius $\rho(K)<1$ such that
\begin{equation}
|F(x) - x^*|\leq K|x - x^*|, \quad \forall x \in \mathbb{R}^n.
\end{equation}
    
\end{frame}

\begin{frame}{Lemma - Perturbed nonnegative matrix}
\begin{lemma}\label{lm: perturbed}
Let $A$ be a $n$-dimensional nonnegative square matrix with spectral radius $\rho(A)<1$. Then there exists a strictly positive matrix $B$ such that 
$$
A< B\text{ and $\rho(B)<1$}
$$
\end{lemma}
\textbf{Remark}: One way to show this is via eigenvalue is continuous function of the matrix. But I prove this lemma using Gelfand's formula.
    
\end{frame}
\begin{frame}{Gelfand's formula}
\begin{lemma}[Gelfand's formula]\label{lm:gelfand}
    If $B$ is any square matrix and $\|\cdot\|$ is any matrix norm, then
    $$
    \rho(B)^k\le \|B^k\|\quad \text{for all $k\in\mathbb{N}$}
    $$
    $$
    \|B^k\|^{1/k}\to \rho(B)\text{ as $k\to\infty$}
    $$
\end{lemma}
\begin{corollary}\label{col:gelfand col}
    If $B$ is any square matrix and $\|\cdot\|$ is any matrix form, then if there exists $n\in\mathbb{N}$ such that 
    $$
    \|B^n\|<1
    $$
    this implies $\rho(B)<1$.
\end{corollary}
    
\end{frame}

\begin{frame}{Proof of the lemma}
Let $J$ denote the $n$-dimensional square matrix with every entry equals to 1. We construct $B = A +\epsilon  J$. We show that there exists $0<\epsilon<1$ such that $\rho(B)<1$.\\
    \\
    Using the Gelfand's formula, we have there exists $N\in\mathbb{N}$ such that for all $n\ge N$, $\|A^n\|< 1$. Fix $n\ge N$. We set $\delta:= 1-\|A^n\|$.\\
    \\
    Moreover, we have 
    \begin{align*}
        \|B^n\| &= \|(A+\epsilon J)^n\|\\
        &= \|A^n +  \epsilon (\Gamma_{1,1} +\cdots + \Gamma_{1,C^n_1})+\cdots +  \epsilon^{n-1}(\Gamma_{n-1, 1} + \cdots \Gamma_{n-1, C^{n}_{n-1}}) +\epsilon^n J^n\|
    \end{align*}
    for some square matrix $\Gamma_{i,j}$ and $C^i_j$ be the number of combinations of choosing $j$ objects from $i$ objects.
    
\end{frame}
\begin{frame}{Remark on the expansion}
Moreover, we have 
    \begin{align*}
        \|B^n\| &= \|(A+\epsilon J)^n\|\\
        &= \|A^n +  \epsilon (\Gamma_{1,1} +\cdots + \Gamma_{1,C^n_1})+\cdots +  \epsilon^{n-1}(\Gamma_{n-1, 1} + \cdots \Gamma_{n-1, C^{n}_{n-1}}) +\epsilon^n J^n\|
    \end{align*}
    for some square matrix $\Gamma_{i,j}$ and $C^i_j$ be the number of combinations of choosing $j$ objects from $i$ objects.
To motive this step, we have for $n=2$,
    \begin{align*}
        (A + \epsilon J)^2  &= A^2 + \epsilon AJ + \epsilon J A + \epsilon^2 J^2\\
        &= A^2 + \epsilon (AJ +JA) +\epsilon ^2 J^2
    \end{align*}
    Hence, we have $\Gamma_{1,1} = AJ$ and $\Gamma_{1,2} = JA$ with $C^2_1 =2$.
\end{frame}
\begin{frame}{Proof}
Moreover, we have 
    \begin{align*}
        \|B^n\| &= \|(A+\epsilon J)^n\|\\
        &= \|A^n +  \epsilon (\Gamma_{1,1} +\cdots + \Gamma_{1,C^n_1})+\cdots +  \epsilon^{n-1}(\Gamma_{n-1, 1} + \cdots \Gamma_{n-1, C^{n}_{n-1}}) +\epsilon^n J^n\|
    \end{align*}
    for some square matrix $\Gamma_{i,j}$ and $C^i_j$ be the number of combinations of choosing $j$ objects from $i$ objects.
\noindent Then by triangle inequality, we have
    \begin{align*}
         \|B^n\| &\le \|A^n\| + \sum_{k=1}^{n-1} \epsilon^k \left(\sum_{j=1}^{C^n_k} \|\Gamma_{k,j}\|\right) + \epsilon^n \|J^n\|
    \end{align*}
    
\end{frame}
\begin{frame}{Proof}
Let 
    $$
    M:=\max_{1\le k,j\le n}\{\|\Gamma_{k,j}\|, \|J^n\|\}
    $$
    $$
    \gamma: =\max_{1\le k\le n} C^n_k
    $$
    By finite dimension, we have $M$ and $\gamma$ is well-defined and finite. This gives
   \begin{align*}
         \|B^n\| &\le \|A^n\| + \gamma M\sum_{k=1}^{n} \epsilon^k\\
         &<\|A^n\| + \gamma M n\epsilon \tag{$0<\epsilon<1$}
    \end{align*}
    Let $0<\epsilon <\frac{\delta}{\gamma Mn}$. Then, we have
    $$
    \|B^n\| = \|(A+\epsilon J)^n\|< \|A^n\| + \delta < 1
    $$
    By the previous corollary, this implies $\rho(B)<1$.
    
\end{frame}
\begin{frame}{Main extension proof - Eventual contraction implies contraction with a specific weighted maximum norm}
Suppose there exists a vector $x^*\in\mathbb{R}^n$ and a positive linear operator $K$ with spectral radius $\rho(K)<1$ such that 
$$
|F(x)-x^*|\le K|x-x^*|,\quad \forall x\in \mathbb{R}^n
$$
Then, this implies there exists a positive vector $v\in\mathbb{R}^n$ and a scalar $\beta\in [0,1)$, such that
$$
\|F(x)-x^*\|_v \le \beta\|x-x^*\|_v
$$
In other words, eventual contraction assumption implies contraction assumption.
    
\end{frame}
\begin{frame}{Proof}
First, since $K$ is a positive linear operator in a finite dimensional space, it can be represented by a nonnegative matrix with spectral radius $\rho(K)<1$.\\
    \\
    By lemma on perturbed nonnegative matrix, there exists a strictly positive matrix $\tilde K>K$ such that $\rho(\tilde K)<1$.\\
    \\
    Using the Perron-Frobenius theorem, we know
    \begin{itemize}
        \item the spectral radius $\beta:=\rho(\tilde K) =\frac{(\tilde Kv)_i}{v_i}<1 $ is a positive real simple eigenvalue of $\tilde K$
        \item Its corresponding eigenvector $v$ is uniquely positive up to positive scaling. 
    \end{itemize}
    
\end{frame}
\begin{frame}{Proof}
Hence, we have pointwise 
    $$
    |F_i(x)-x^*_i|\le (K|x-x^*|)_i \le (\tilde K|x-x^*|)_i,\quad i=1,2,\cdots, n
    $$
    as $K<\tilde K$. Using the matrix representation, we have
    $$
    (\tilde K|x-x^*|)_i= \sum_{j=1}^n \tilde K_{ij}|x_j-x_j^*|
    $$
    We define
    $$
    \|z\|_v:=\max_{1\le i\le n}\frac{|z_i|}{v_i},\quad \forall z\in \mathbb{R}^n
    $$
    as the weighted maximum norm using $v$. Hence, this implies
    $$
    \frac{|z_j|}{v_j}\le \max_{1\le i\le n}\frac{|z_i|}{v_i},\quad j=1,2,\cdots,n
    $$
    
\end{frame}
\begin{frame}{proof}
    Hence,
    $$
    |z_j|\le v_j\|z\|_v,\quad j=1,2,\cdots,n
    $$
    We can apply this to $|x_j-x_j^*|$, we get
    \begin{align*}
        (\tilde K|x-x^*|)_i &= \sum_{j=1}^n \tilde K_{ij}|x_j-x_j^*|\\
        &\le \sum_{j=1}^n \tilde K_{ij} v_j\|x-x^*\|_v\\
        &= \|x-x^*\|_v\sum_{j=1}^n \tilde K_{ij}v_j\\
        &= \|x-x^*\|_v(\tilde Kv)_i\\
    \end{align*}
    
\end{frame}

\begin{frame}{proof}
This implies
    $$
    |F_i(x)-x^*_i|\le\|x-x^*\|_v(\tilde Kv)_i
    $$
    Now we divide both sides by $v_i$, we get
    \begin{align*}
        \frac{|F_i(x)-x^*_i|}{v_i}\le \frac{(\tilde Kv)_i}{v_i} \|x-x^*\|_v= \beta \|x-x^*\|_v
    \end{align*}
    for all $i=1,2,\cdots, n$. Hence, we have
    $$
    \|F(x)-x^*\|_v = \max_{1\le i \le n}\frac{|F_i(x)-x^*_i|}{v_i} \le\beta \|x-x^*\|_v
    $$
    This completes the proof.
    
\end{frame}
\begin{frame}{Appendix - direct comparison with Tsitsiklis 1994 setup}
    We consider iterative updates of a vector $x \in \mathbb{R}^n$ to solve the fixed-point equation $F(x^*) = x^*$, where $F: \mathbb{R}^n \mapsto \mathbb{R}^n$ with component mappings $F_i: \mathbb{R}^n \mapsto \mathbb{R}$.\\
\\
Let $x(t)$ denote the state at discrete time $t \in \mathbb{N}$, with components $x_i(t)$. For each component $i$, we have:
\begin{equation}\label{eq:update_rule_tsi}
x_i(t + 1) = 
\begin{cases}
x_i(t), & t \notin T^i \\
x_i(t) + \alpha_i(t)(F_i(x^i(t)) - x_i(t) + w_i(t)), & t \in T^i
\end{cases}
\end{equation}
where:
\begin{itemize}
\item $T^i \subset \mathbb{N}$ is the set of update times for component $i$
\item $\alpha_i(t) \in [0,1]$ is the stepsize parameter
\item $w_i(t)$ is a noise term
\item $x^i(t) = (x_1(\tau_1^i(t)), \ldots, x_n(\tau_n^i(t)))$ contains possibly outdated information with $0 \leq \tau_j^i(t) \leq t$
\end{itemize}
All variables are defined on a probability space $(\Omega, \mathcal{F}, P)$ with an increasing sequence of $\sigma$-fields $\{\mathcal{F}(t)\}_{t=0}^{\infty}$ representing the algorithm's history.
\end{frame}
\begin{frame}{Simplified setup notation}
Let $x(t)$ denote the state at discrete time $t\in\mathbb{N}$ with component $x_i(t)$. For each component, we have
\begin{equation}
    x_i(t+1) = (1-\alpha_i(t))x_i(t) + \alpha_i(t)(F_i(\textcolor{blue}{x(t)}) + w_i(t)) 
\end{equation}\label{eq:x_t}


    $$
    \mathbf{x}(t+1) =(I-\mathbf{A}(t)) \mathbf{x}(t) + \mathbf{A}(t)(\mathbf{F}(\mathbf{x}(t)) +\mathbf{w}(t))
    $$
    where
    $$
    \mathbf{x}(t) = \begin{pmatrix}
        x_1(t)\\
        \vdots\\
        x_n(t)
    \end{pmatrix},\mathbf{A}(t) = \begin{pmatrix}
        \alpha_1(t) &  \cdots & 0\\
        \vdots  & \ddots & 0\\
        0 &\cdots & \alpha_n(t)
    \end{pmatrix},\mathbf{F}(\mathbf{x}(t)) = \begin{pmatrix}
        F_1(\mathbf{x}(t))\\
        \vdots\\
        F_n(\mathbf{x}(t))
    \end{pmatrix},\mathbf{w}(t) = \begin{pmatrix}
        w_1(t)\\
        \vdots\\
        w_n(t)
    \end{pmatrix}
    $$
\end{frame}
\begin{frame}{Martingale, sub- and super-martingale}
\begin{definition}
    Let $(\Omega, \mathcal{F}, \mathbb{P})$ be a probability space with filtration $\mathbb{F} =(\mathcal{F}(t))_{t\ge 0}$. A stochastic process $X = (X(t))_{t\ge 0}$ is called a martingale with respect to the filtration $\mathbb{F}$ if
    \begin{enumerate}
        \item $X$ is adaped to $\mathbb{F}$
        \item $\mathbb{E}_{\mathbb{P}} |X(t)|<\infty$ for all $t\ge 0$
        \item For $s\le t$, $\mathbb{E}_{\mathbb{P}} (X(t)|\mathcal{F}(s)) = X(s)$
    \end{enumerate}
    A stochastic process $X(t)$ is called a submartingale if the third condition becomes
    $$
    s\le t, \mathbb{E}_{\mathbb{P}} (X(t)|\mathcal{F}(s))\ge X(s)
    $$
    A stochastic process $X(t)$ is called a supermartingale if the third condition becomes
    $$
    s\le t, \mathbb{E}_{\mathbb{P}} (X(t)|\mathcal{F}(s))\le X(s)
    $$
\end{definition}
    
\end{frame}
\begin{frame}{Full proof for Lemma 1}
    \begin{proof}
        Let us first note that, without loss of generality, we can assume that $B(t) \leq K$ for some constant $K$ almost surely, since the sequence $\{B(t)\}$ is bounded with probability 1.\\
\\
\textbf{Step 1: Use the squared process}\\
\\
We analyze the evolution of the squared process $V(t) = W^2(t)$. From the recursion for $W(t)$, we have:
\begin{align*}
W(t+1) &= (1 - \alpha(t))W(t) + \alpha(t)w(t)
\end{align*}

\noindent Squaring both sides yields:
\begin{align*}
W^2(t+1) &= \left((1 - \alpha(t))W(t) + \alpha(t)w(t)\right)^2 \\
&= (1 - \alpha(t))^2W^2(t) + 2(1-\alpha(t))\alpha(t)W(t)w(t) + \alpha^2(t)w^2(t)
\end{align*}
    \end{proof}
\end{frame}
\begin{frame}{Full proof for lemma 1 part 2}
\begin{proof}
\noindent Taking the conditional expectation with respect to $\mathcal{F}(t)$:
\begin{align*}
\mathbb{E}[W^2(t+1) \mid \mathcal{F}(t)] &= (1 - \alpha(t))^2W^2(t) + 2(1-\alpha(t))\alpha(t)W(t)\mathbb{E}[w(t) \mid \mathcal{F}(t)] + \alpha^2(t)\mathbb{E}[w^2(t) \mid \mathcal{F}(t)]
\end{align*}
\noindent Using the conditions $\mathbb{E}[w(t) \mid \mathcal{F}(t)] = 0$ and $\mathbb{E}[w^2(t) \mid \mathcal{F}(t)] \leq B(t)\le K$, we obtain:
\begin{align*}
\mathbb{E}[V(t+1) \mid \mathcal{F}(t)] &\leq (1 - \alpha(t))^2V(t) + \alpha^2(t)K \\
&= (1 - 2\alpha(t) + \alpha^2(t))V(t) + \alpha^2(t)K \\
&= V(t) - 2\alpha(t)V(t) + \alpha^2(t)V(t) + \alpha^2(t)K \\
&= V(t) - \alpha(t)V(t)(2 - \alpha(t)) + \alpha^2(t)K
\end{align*}

\end{proof}
    
\end{frame}
\begin{frame}{Full proof of lemma 1}
\begin{proof}
    \noindent Since $\alpha(t) \in [0,1]$, we have $(2 - \alpha(t)) \geq 1$, which gives:
\begin{align*}
\mathbb{E}[V(t+1) \mid \mathcal{F}(t)] &\leq V(t) - \alpha(t)V(t) + \alpha^2(t)K \\
&= (1 - \alpha(t))V(t) + \alpha^2(t)K\\
&= V(t) + \alpha^2(t) K -\alpha(t) V(t)
\end{align*}
\end{proof}
    
\end{frame}
\begin{frame}{Full proof of lemma 1}
\begin{proof}
    \textbf{Step 2: Use \autoref{thm:robbins siegmund}}\\
\\
\noindent Now, we let
\begin{itemize}
    \item $\xi_t = \alpha^2(t)K$, we have 
    $$
    \sum_{t=0}^\infty \xi_t  = \sum_{t=0}^\infty \alpha(t)^2K = K\sum_{t=0}^\infty \alpha^2(t) <\infty
    $$
    by our assumption. 
    \item $\zeta_t = \alpha(t)V(t)$ is nonnegative and adapted to the filtration. 
\end{itemize}
Hence, we use \autoref{thm:robbins siegmund}, we get
\begin{itemize}
    \item $\lim_{t\to\infty} V(t) = V_\infty$ exists and is finite almost surely
    \item $\sum_{t=0}^\infty \zeta_t = \sum_{t=0}^\infty \alpha(t) V(t)<\infty$ almost surely.
\end{itemize}
\end{proof}
\end{frame}
\begin{frame}{Full proof of lemma 1}
\begin{proof}
    \textbf{Step 3: Prove $V_\infty =0$ almost surely by contradiction}\\
\\
Suppose that $P(V_\infty\ge 2\epsilon)>\delta$ for some $\epsilon, \delta>0$. Then we have on the set $\{\omega: V_\infty(\omega)\ge 2\epsilon\}$, by the definition of limit, for every $\omega\in \{\omega: V_\infty(\omega)\ge 2\epsilon\}$, there exists $T(\omega)\in\mathbb{N}$ such that for all $t\ge T(\omega)$, $V(t,\omega)\ge \epsilon$. Hence for all $\omega\in\{V_\infty\ge \epsilon\}$:
$$
\sum_{t=0}^\infty \zeta_t(\omega) = \sum_{t=0}^\infty \alpha(t)V(t,\omega)\ge \sum_{t=T(\omega)}^\infty \alpha(t) V(t,\omega)\ge \epsilon\sum_{t=T(\omega)}^\infty \alpha (t)
$$
By $\sum_{t=0}^\infty \alpha (t) = \infty$, we have $\sum_{t=T(\omega)}^\infty \alpha(t) = \infty$. Hence
$$
\sum_{t=0}^\infty \zeta_t(\omega) \ge \epsilon\sum_{t=T(\omega)}^\infty \alpha (t) = \infty
$$
\end{proof}
\end{frame}
\begin{frame}{Full proof of lemma 1}
\begin{proof}
    This implies 
$$
\left\{\omega: \sum_{t=0}^\infty \zeta_t(\omega)=\infty\right\}\supseteq \{\omega: V_\infty(\omega)\ge 2\epsilon\}
$$
Hence
$$
P\left(\sum_{t=0}^\infty \zeta_t =  \infty\right) \ge P\left(V_\infty \ge 2\epsilon\right)>\delta
$$
This contradicts to $\sum_{t=0}^\infty \zeta_t<\infty$ almost surely.\\
\\
Hence, this contradiction gives $V_\infty =0$ almost surely.
\end{proof}
    
\end{frame}
\end{document}
