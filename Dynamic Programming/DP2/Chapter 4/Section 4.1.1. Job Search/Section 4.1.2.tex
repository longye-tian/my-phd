\section{Rearranging the Bellman Equation}
\begin{frame}{Section 4.1.2 Rearranging the Bellman Equation}
    \begin{itemize}
        \item Rearranging the Bellman Equation
    \end{itemize}
\end{frame}
\begin{frame}{Rearranging the Bellman Equation - Continuation Value}
    Given the Bellman equation
    $$
    v(w)  = \max\left\{\frac{w}{1-\beta}, \textcolor{blue}{c+\beta\int v(w') \,\varphi(dw')}\right\}\quad\text{for all $w\in W$}
    $$
    Let 
    $$
    h:=\textcolor{blue}{c+\beta\int v(w') \,\varphi(dw')}
    $$
    be the continuation value. And $h\in \mathbb{R}$.
\end{frame}

\begin{frame}{Continuation value}
We get a one-dimensional nonlinear equation. 
    \begin{align*}
        h &=\textcolor{blue}{c+\beta\int v(w') \,\varphi(dw')}\\
        &= c + \beta \int \max\left\{\frac{w'}{1-\beta}, \textcolor{blue}{c+\beta\int v(w'') \,\varphi(dw'')}\right\}\, \varphi(dw')\\
        &= c+\beta\int \max\left\{\frac{w'}{1-\beta}, h\right\}\, \varphi(dw')
    \end{align*}
\end{frame}

\begin{frame}{Exercise 4.1.5}
    We introduce
    $$
    g(h) = c+\beta\int \max\left\{\frac{w'}{1-\beta}, h\right\}\, \varphi(dw')
    $$
    and we want to find the fixed point of $g$. Using the elementary bound
    $$
    |\alpha\vee x- \alpha \vee y|\le |x-y|
    $$
    show that $g$ is a contraction.
\end{frame}
\begin{frame}{Exercise 4.1.5 cont.}
\begin{proof}
    Let $h,k\in\mathbb{R}_+$. we have
    \begin{align*}
        |g(h)-g(k)| &= \beta\left|\int \max\left\{\frac{w'}{1-\beta},h\right\}\,\varphi(dw')-\int \max\left\{\frac{w'}{1-\beta},k\right\}\,\varphi(dw')\right|\\
        &\le \beta \int\left|\max\left\{\frac{w'}{1-\beta},h\right\}-\max\left\{\frac{w'}{1-\beta},g\right\}\right|\,\varphi(dw')\tag{Jensen}\\
        &\le \beta \int |h-k|\,\varphi(dw')\tag{Given}\\
        &= \beta|h-k|
    \end{align*}
\end{proof}
 Hence we can get a unique fixed point. Optimal policy is hence
    $$
    \sigma_\top (w) = \mathbf{1}\{w\ge w_\top\},\quad \text{where $w_\top:= (1-\beta)h_\top$}
    $$
\end{frame}

\begin{frame}{Reduce the search space further}
    Let $f(h):= c+\beta\bar w/(1-\beta) + \beta h$. We can find the fixed point $h^*$ as
    $$
    h^* = c + \frac{\beta}{1-\beta}\int w\,\varphi(dw) + \beta h^*
    $$
    We have 
    \begin{align*}
        g(h^*) & = c+ \beta \int \max\left\{\frac{w'}{1-\beta}, h^*\right\}\,\varphi(dw')\\
        &= c+ \beta\left(\int_{\{w'/(1-\beta)\ge h^*\}}\frac{w'}{1-\beta}\, \varphi(dw') + \int_{\{w'/(1-\beta)\le h^*\}}h^*\, \varphi(dw')\right)\\
        &= c + \frac{\beta}{1-\beta}\int_{\{w/(1-\beta)\ge h^*\}}w\,\varphi(dw) + \beta h^*\varphi(\{w/(1-\beta)\le h^*\})
    \end{align*}
\end{frame}

\begin{frame}{Reduce the search space continued}

$$
 h^* = c + \frac{\beta}{1-\beta}\int w\,\varphi(dw) + \beta h^*
$$

$$
 g(h^*) =c + \frac{\beta}{1-\beta}\int_{\underbrace{\{w/(1-\beta)\ge h^*\}}_{\varphi(\cdot)\le1}}w\,\varphi(dw) + \beta h^*\underbrace{\varphi(\{w/(1-\beta)\le h^*\})}_{\le1}\le h^*
$$
Hence, $g$ maps $h^*$ down.\\
\\
$g$ is a contraction $\implies$ globally stable $\implies$ strongly order stable $\implies$ $g$ is a self-map on $[0, h^*]$.
\end{frame}

\begin{frame}{Parametric Monotonicity}
    \begin{itemize}
        \item parameters play a key role in dynamics
        \item Is the solution robust to parameter change?
        \item  How does the solution vary with parameters?
        \item Useful to robustness check
        \item useful to policy design
    \end{itemize}
\end{frame}

\begin{frame}{Parametric Monotonicity}
        Given the Bellman equation
    $$
    v(w)  = \max\left\{\frac{w}{1-\beta}, c+\beta\int v(w') \,\varphi(dw')\right\}\quad\text{for all $w\in W$}
    $$
    How does changes in 
    \begin{itemize}
        \item unemployment compensation $c$
        \item discount factor $\beta$
        \item distribution $\varphi$
    \end{itemize}
    change the reservation wage $w_\top = (1-\beta) (c+\beta\int v_\top(w')\, \varphi(dw'))$.
\end{frame}

\begin{frame}{Proposition A.5.18}
Let $V$ be a pospace and $S,T$ be two self-map on $V$ ordered pointwise, i.e.,
$$
S\precsim T\iff Sv\precsim Tv\quad \text{for all $v\in V$}
$$
\textbf{Proposition A.5.18}
    If $S\precsim T$, $T$ is \textcolor{blue}{order preserving and globally stable} on $V$, then its unique fixed points dominates any fixed point of $S$.
\begin{proof}
   $$
   v_S = Sv_S\precsim Tv_S\implies v_S\precsim v_T
   $$
   by (strongly) order stability from global stability + order preserving.
\end{proof}
    
\end{frame}

\begin{frame}{Exercise 4.1.8}
Prove that the reservation wage $w_\top$ is increasing in unemployment compensation $c$. Prove also that $h_\top$ is increasing with $\beta$.
\begin{proof}
    We have for $c_1\le c_2$
    $$
    g_1(h) = c_1 + \beta \int \max\left\{\frac{w'}{1-\beta}, h\right\}\,\varphi(dw')\le c_2 + \beta \int \max\left\{\frac{w'}{1-\beta}, h\right\}\,\varphi(dw') = g_2(h)
    $$
\begin{itemize}
    \item By \textcolor{blue}{PropA518}, $h^1_\top \le h^2_\top$.
    \item By $w_\top = (1-\beta)h_\top$, $w^1_\top\le w^2_\top$
\end{itemize}
Higher unemployment compensation, higher reservation wage.
\end{proof}


\end{frame}

\begin{frame}{Exercise 4.1.0}
Let $\tau$ be the first passage time to employment, i.e.,
$$
\tau:= \inf\{t\ge 0: \sigma_\top (W_t) = 1\} = \inf\{t\ge 0: W_t\ge w_\top\}
$$
Prove that the mean first passage time $\mathbb{E}\tau$ increases with $c$.

\textbf{Remark:} Here $\tau$ is a random variable with sample space $\Omega = W^\mathbb{N}$. We have
\begin{align*}
    \mathbb{E}\tau  &= 0\cdot \mathbb{P}(W_0\ge w_\top) + 1\cdot \mathbb{P}(W_1\ge w_\top
    |W_0< w_\top)+ 2\cdot \mathbb{P}(W_2\ge w_\top|W_0,W_1<w_\top)+\cdots\\
    &= 0\cdot p + 1\cdot p(1-p) + 2\cdot p(1-p)^2 + \cdots\\
    &= \sum_{i=1}^\infty ip(1-p)^i\tag{mean of Geometric distribution}\\
    &=\frac{1-p}{p}
\end{align*}
\end{frame}

\begin{frame}{Higher unemployment compensation, wait longer on average}
    We have $c_1\le c_2\implies w_\top^1\le w_\top^2$, this implies
    $$
    p_1 = \mathbb{P}(W_i\ge w^1_\top)\ge \mathbb{P}(W_i\ge w_\top^2) = p_2
    $$
    $$
    p_1\ge p_2\implies\mathbb{E}\tau|c=c_1 = \frac{1-p_1}{p_1} \le  \frac{1-p_2}{p_2} = \mathbb{E}\tau|c=c_2
    $$
\end{frame}

\begin{frame}{Increase in discount factor}
From 
$$
h_\top = c+\beta\int\max\left\{\frac{w'}{1-\beta}, h_\top\right\}\, \varphi(dw')
$$
We use $w_\top = (1-\beta)h_\top$ get
$$
w_\top  = c(1-\beta)  + \beta\int\max\{w', w_\top\}\,\varphi(dw')
$$
We define the following function (\textcolor{blue}{order-preserving and contraction}):
$$
f(w) =  c(1-\beta)  + \beta\int\max\{w', w\}\,\varphi(dw')
$$
And $w_\top$ is the unique fixed point. 
\end{frame}

\begin{frame}{Increase in discount factor}
We can take partial derivative of $f$ with respect to $\beta$
$$
\frac{\partial f(w)}{\partial \beta} = -c+\int\max\{w',w\}\,\varphi(dw')
$$
Hence, when 
$$
c\le \int w'\,\varphi(dw') \le \int\max\{w',w\}\,d(w')\quad \text{for all $w\in W$}
$$
We have for $\beta_1\le\beta_2$
$$
\textcolor{blue}{f(w;\beta_1)\le f(w;\beta_2)\quad \text{for all $w\in W$}}
$$
By \textcolor{blue}{PropA518}, we have $w_\top^1\le w_\top^2$.
\end{frame}



\begin{frame}{Changes in distribution (Lemma 4.1.2.)}
\begin{lemma}
    If $\varphi,\psi$ are supported on $[0,M]$, $M\in\mathbb{R_+}$, $\psi$ first order stochastically dominates $\varphi$, then $w_\top^\varphi\le w_\top^\psi$.
\end{lemma}
\begin{definition}
    Let $ibX$ be the increasing bounded  functions on $X$. We say that $\nu$ first order stochastically dominates $\mu$ and write $\mu\precsim_F \nu$ if 
    $$
    \int u(x)\,\mu(dx)\le \int u(x)\,\nu(dx)\quad\text{for every $u\in ibX$}
    $$
\end{definition}
    
\end{frame}
\begin{frame}{Remark: $f:\mathbb{R}\to\mathbb{R}$ and increasing implies measurable}
    \begin{proof}
        To prove that $f$ is measurable, we need to show that for any $a \in \mathbb{R}$, the set $f^{-1}((-\infty, a))$ is measurable.
Since $f$ is increasing, for any $a \in \mathbb{R}$, we can define $c = \sup{x \in \mathbb{R} : f(x) < a}$.
By the properties of increasing functions, the set $f^{-1}((-\infty, a))$ takes one of two forms:
\begin{itemize}
\item If $f(c) < a$ (or if $c = \infty$), then $f^{-1}((-\infty, a)) = (-\infty, c]$
\item If $f(c) \geq a$ (or if $c = -\infty$), then $f^{-1}((-\infty, a)) = (-\infty, c)$
\end{itemize}
In either case, the preimage is a Borel set and therefore measurable. Since the preimage of every interval of the form $(-\infty, a)$ is measurable, and these intervals generate the Borel $\sigma$-algebra on $\mathbb{R}$, the function $f$ is measurable.
    \end{proof}
\end{frame}

\begin{frame}{Change in distribution cont.}
\begin{proof}
     We have for $\varphi\precsim_F \psi$, fix $w\in\mathbb{R}^+$,$w'\mapsto \max\{w',w\}\in ib[0,M]$.
     \begin{align*}
         f_\varphi(w) &=  c(1-\beta)  + \beta\textcolor{blue}{\int\max\{w', w\}\,\varphi(dw')}\\
         &\le c(1-\beta)  + \beta\textcolor{blue}{\int\max\{w', w\}\,\psi(dw')}\\
         &=f_\psi(w)
     \end{align*}
    \begin{itemize}
        \item $f_\psi$ is order-preserving and globally stable
        \item $f_\varphi\precsim f_\psi$
    \end{itemize}
    By \textcolor{blue}{PropA518}, $w_\top^\varphi \le w_\top^\psi$
\end{proof}
    
    
\end{frame}




\begin{frame}{Mean-preserving spread}
    We also concern with how \textcolor{blue}{behavior changes when decisions become riskier}. We introduce the notion of \textbf{mean-preserving spread}.
    
    \begin{definition}
        For a given distribution $\varphi$, we say that $\psi$ is a mean-preserving spread of $\varphi$ if there exists a pair of random variable $(Y,Z)$ such that
    $$
    \mathbb{E}[Z|Y] = 0,\quad Y=^d \varphi, \quad Y+Z=^d\psi 
    $$
    \end{definition}
\end{frame}

\begin{frame}{Lemma 4.1.3}
If $\psi$ is a mean-preserving spread of $\varphi$, then $w_\top^\varphi\le w_\top^\psi$.\\
\\
\begin{proof}
    We know that
$$
W_t =^d \varphi,\quad\text{by iid, we have $W_\cdot =^d \varphi$}
$$
Let $\psi$ be a mean-preserving spread of $\varphi$. Then there exists a pair of random variable $(W_\cdot,Z)$ such that
$$
\mathbb{E}[Z|W] = 0,\quad W_\cdot=^d\varphi, \quad W_\cdot+Z=^d\psi
$$
\end{proof}
\end{frame}
\begin{frame}{Lemma 4.1.3 Cont.}
\begin{proof}
    This implies
\begin{align*}
    \int\max\{w',w\}\,\psi(dw') &= \mathbb{E}[\max\{W_\cdot+Z, w\}]= \mathbb{E}\bigg[\textcolor{blue}{\mathbb{E}\left[\textcolor{red}{\max\{}W_\cdot+Z, \textcolor{red}{w\}}|W_\cdot\right]}\bigg]\tag{LIE}\\
    &\ge \mathbb{E}\bigg[\textcolor{red}{\max\{}\textcolor{blue}{\mathbb{E}[W_\cdot+Z|W_\cdot]},\textcolor{red}{w\}}\bigg]\tag{Cond. Jensen}\\
    &= \mathbb{E}\bigg[\max\{W_\cdot,w\}\bigg]\tag{Linearity}\\
    &= \int\max\{w',w\}\,\varphi(dw')
\end{align*}
\end{proof}
    
\end{frame}