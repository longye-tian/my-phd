\section{Rearranging the Bellman Equation}
\begin{frame}{Rearranging the Bellman Equation - Continuation Value}
    Given the Bellman equation
    $$
    v(w)  = \max\left\{\frac{w}{1-\beta}, \textcolor{blue}{c+\beta\int v(w') \,\varphi(dw')}\right\}\quad\text{for all $w\in W$}
    $$
    Let 
    $$
    h:=\textcolor{blue}{c+\beta\int v(w') \,\varphi(dw')}
    $$
    be the continuation value. And $h\in \mathbb{R}$.
\end{frame}

\begin{frame}{Continuation value}
We get a one-dimensional nonlinear equation. 
    \begin{align*}
        h &=\textcolor{blue}{c+\beta\int v(w') \,\varphi(dw')}\\
        &= c + \beta \int \max\left\{\frac{w'}{1-\beta}, \textcolor{blue}{c+\beta\int v(w'') \,\varphi(dw'')}\right\}\, \varphi(dw')\\
        &= c+\beta\int \max\left\{\frac{w'}{1-\beta}, h\right\}\, \varphi(dw')
    \end{align*}
\end{frame}

\begin{frame}{Solve the equation by finding a fixed point}
    We introduce
    $$
    g(h) = c+\beta\int \max\left\{\frac{w'}{1-\beta}, h\right\}\, \varphi(dw')
    $$
    and we want to find the fixed point of $g$. Using the following inequality
    $$
    |\alpha\vee x- \alpha \vee y|\le |x-y|
    $$
    we can show that $g$ is a contraction. Hence we can get a unique fixed point. Optimal policy is hence
    $$
    \sigma_\top (w) = \mathbf{1}\{w\ge w_\top\},\quad \text{where $w_\top:= (1-\beta)h_\top$}
    $$
\end{frame}

\begin{frame}{Reduce the search space further}
    Let $f(h):= c+\beta\bar w/(1-\beta) + \beta h$. We can find the fixed point $h^*$ as
    $$
    h^* = c + \frac{\beta}{1-\beta}\int w\,\varphi(dw) + \beta h^*
    $$
    We have 
    \begin{align*}
        g(h^*) & = c+ \beta \int \max\left\{\frac{w'}{1-\beta}, h^*\right\}\,\varphi(dw')\\
        &= c+ \beta\left(\int_{\{w'/(1-\beta)\ge h^*\}}\frac{w'}{1-\beta}\, \varphi(dw') + \int_{\{w'/(1-\beta)\le h^*\}}h^*\, \varphi(dw')\right)\\
        &= c + \frac{\beta}{1-\beta}\int_{\{w/(1-\beta)\ge h^*\}}w\,\varphi(dw) + \beta h^*\varphi(\{w/(1-\beta)\le h^*\})
    \end{align*}
\end{frame}

\begin{frame}{Reduce the search space continued}

$$
 h^* = c + \frac{\beta}{1-\beta}\int w\,\varphi(dw) + \beta h^*
$$

$$
 g(h^*) =c + \frac{\beta}{1-\beta}\int_{\underbrace{\{w/(1-\beta)\ge h^*\}}_{\varphi(\cdot)\le1}}w\,\varphi(dw) + \beta h^*\underbrace{\varphi(\{w/(1-\beta)\le h^*\})}_{\le1}\le h^*
$$
Hence, $g$ maps $h^*$ down.\\
\\
$g$ is a contraction $\implies$ globally stable $\implies$ strongly order stable $\implies$ $g$ is a self-map on $[0, h^*]$.
\end{frame}

\begin{frame}{Parametric Monotonicity}
    \begin{itemize}
        \item parameters play a key role in dynamics
        \item Is the solution robust to parameter change?
        \item  How does the solution vary with parameters?
        \item Useful to robustness check
        \item useful to policy design
    \end{itemize}
\end{frame}

\begin{frame}{Parametric Monotonicity}
        Given the Bellman equation
    $$
    v(w)  = \max\left\{\frac{w}{1-\beta}, c+\beta\int v(w') \,\varphi(dw')\right\}\quad\text{for all $w\in W$}
    $$
    How does changes in 
    \begin{itemize}
        \item unemployment compensation $c$
        \item discount factor $\beta$
        \item distribution $\varphi$
    \end{itemize}
    change the reservation wage $w_\top = (1-\beta) (c+\beta\int v_\top(w')\, \varphi(dw'))$.
\end{frame}

\begin{frame}{Proposition A.5.18}
Let $V$ be a pospace and $S,T$ be two self-map on $V$ ordered pointwise, i.e.,
$$
S\precsim T\iff Sv\precsim Tv\quad \text{for all $v\in V$}
$$
If $S\precsim T$, $T$ is order preserving and globally stable on $V$, then its unique fixed points dominates any fixed point of $S$.
\begin{proof}
   $$
   v_S = Sv_S\precsim Tv_S\implies v_S\precsim v_T
   $$
   by (strongly) order stability from global stability + order preserving.
\end{proof}
    
\end{frame}

\begin{frame}{Unemployment Compensation}
    We have for $c_1\le c_2$
    $$
    g_1(h) = c_1 + \beta \int \max\left\{\frac{w'}{1-\beta}, h\right\}\,\varphi(dw')\le c_2 + \beta \int \max\left\{\frac{w'}{1-\beta}, h\right\}\,\varphi(dw') = g_2(h)
    $$
\begin{itemize}
    \item By \textcolor{blue}{PropA518}, $h^1_\top \le h^2_\top$.
    \item By $w_\top = (1-\beta)h_\top$, $w^1_\top\le w^2_\top$
\end{itemize}
Higher unemployment compensation, higher reservation wage.
\end{frame}

\begin{frame}{Higher unemployment compensation, wait longer on average}
Let $\tau$ be the first passage time to employment, i.e.,
$$
\tau:= \inf\{t\ge 0: \sigma_\top (W_t) = 1\} = \inf\{t\ge 0: W_t\ge w_\top\}
$$
\textbf{Remark:} Here $\tau$ is a random variable with sample space $\Omega = W^\mathbb{N}$. We have
\begin{align*}
    \mathbb{E}\tau  &= 0\cdot \mathbb{P}(W_0\ge w_\top) + 1\cdot \mathbb{P}(W_1\ge w_\top
    |W_0< w_\top)+ 2\cdot \mathbb{P}(W_2\ge w_\top|W_0,W_1<w_\top)+\cdots\\
    &= 0\cdot p + 1\cdot p(1-p) + 2\cdot p(1-p)^2 + \cdots\\
    &= \sum_{i=1}^\infty ip(1-p)^i\tag{mean of Geometric distribution}\\
    &=\frac{1-p}{p}
\end{align*}
\end{frame}

\begin{frame}{Higher unemployment compensation, wait longer on average}
    We have $c_1\le c_2\implies w_\top^1\le w_\top^2$, this implies
    $$
    p_1 = \mathbb{P}(W_i\ge w^1_\top)\ge \mathbb{P}(W_i\ge w_\top^2) = p_2
    $$
    $$
    p_1\ge p_2\implies\mathbb{E}\tau|c=c_1 = \frac{1-p_1}{p_1} \le  \frac{1-p_2}{p_2} = \mathbb{E}\tau|c=c_2
    $$
\end{frame}

\begin{frame}{Increase in discount factor}
    
\end{frame}

\begin{frame}{Changes in distribution}
If $\psi$ first order stochastically dominates $\varphi$, then $w_\top^\varphi\le w_\top^\psi$.
\begin{definition}
    Let $ibX$ be the increasing bounded real-value functions on $X$. We say that $\nu$ first order stochastically dominates $\mu$ and write $\mu\precsim_F \nu$ if 
    $$
    \int u(x)\,\mu(dx)\le \int u(x)\,\nu(dx)\quad\text{for every $u\in ibX$}
    $$
\end{definition}
\begin{proof}
     We have for $c_1\le c_2$
    $$
    g_\varphi(h) = c + \beta \int \max\left\{\frac{w'}{1-\beta}, h\right\}\,\varphi(dw')\le c+ \beta \int \max\left\{\frac{w'}{1-\beta}, h\right\}\,\psi(dw') = g_\psi(h)
    $$
\end{proof}
    
\end{frame}


\begin{frame}{Mean-preserving spread}
    
\end{frame}