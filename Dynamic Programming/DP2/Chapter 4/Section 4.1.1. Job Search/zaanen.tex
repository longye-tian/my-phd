%%%%%%%%%%%%%%%%%%%%%%%%%%%%%%%%%%%%%%%%%
% Professional Mathematical Presentation Template
% 
% This template uses the beamer class with the Madrid theme
% and a custom color scheme for a clean, professional look
% that works well with mathematical content.
%%%%%%%%%%%%%%%%%%%%%%%%%%%%%%%%%%%%%%%%%
\documentclass[aspectratio=169]{beamer} % 16:9 aspect ratio (modern)
% Theme settings
\usetheme{Madrid}
\usecolortheme{default}
\usepackage[dvipsnames]{xcolor}
\definecolor{primcolor}{RGB}{25,50,100} % Dark blue
\setbeamercolor{structure}{fg=primcolor}
\setbeamercolor{frametitle}{bg=primcolor!15, fg=primcolor}
\setbeamercolor{title}{fg=white} % White title text for contrast
\setbeamercolor{subtitle}{fg=white} % White subtitle text
\setbeamercolor{author}{fg=primcolor} % White author text
\setbeamercolor{date}{fg=primcolor} % White date text
\setbeamercolor{institute}{fg=primcolor} % White institute text
% Font settings
\usefonttheme{professionalfonts}
\usefonttheme{serif}
% Package imports
\usepackage{amsmath, amsfonts, amssymb, amsthm} % Math packages
\usepackage{mathtools} % Enhanced math tools
\usepackage{bm} % Bold math symbols
\usepackage{graphicx} % For images
\usepackage{booktabs} % Professional tables
\usepackage{tikz} % For diagrams
\usetikzlibrary{arrows, positioning, matrix, decorations.pathreplacing}
% Use beamer's theorem styles
\setbeamertemplate{theorem}[ams style]
\setbeamertemplate{theorems}[numbered]
% Remove navigation symbols
\setbeamertemplate{navigation symbols}{}
% Custom footer
\setbeamertemplate{footline}{
  \leavevmode%
  \hbox{%
  \begin{beamercolorbox}[wd=.333333\paperwidth,ht=2.25ex,dp=1ex,center]{author in head/foot}%
    \usebeamerfont{author in head/foot}\insertshortauthor
  \end{beamercolorbox}%
  \begin{beamercolorbox}[wd=.333333\paperwidth,ht=2.25ex,dp=1ex,center]{title in head/foot}%
    \usebeamerfont{title in head/foot}\insertshorttitle
  \end{beamercolorbox}%
  \begin{beamercolorbox}[wd=.333333\paperwidth,ht=2.25ex,dp=1ex,right]{date in head/foot}%
    \usebeamerfont{date in head/foot}\insertshortdate{}\hspace{2em}
    \insertframenumber{} / \inserttotalframenumber\hspace{2ex} 
  \end{beamercolorbox}}%
  \vskip0pt%
}
% Title information
\title[Zaanen]{Every positive operator on $L^p$ is order continuous}
\subtitle{by Adriaan C. Zaanen}
\author[Longye]{Longye Tian \\ \texttt{longye.tian@anu.edu.au}}
\institute[ANU]{Australian National University\\ School of Economics}
\date{April 17th, 2025}
\DeclareFontFamily{U}{mathx}{\hyphenchar\font45}
\DeclareFontShape{U}{mathx}{m}{n}{
      <5> <6> <7> <8> <9> <10>
      <10.95> <12> <14.4> <17.28> <20.74> <24.88>
      mathx10
      }{}
\DeclareSymbolFont{mathx}{U}{mathx}{m}{n}
\DeclareMathSymbol{\bigtimes}{1}{mathx}{"91}

\begin{document}

% Title frame
\begin{frame}
  \titlepage
\end{frame}

% Outline frame
\begin{frame}{Outline}
  \begin{enumerate}
  \item Definitions
  \item Exercise 17.15
  \item Theorem 18.4
  \item Example 21.6
  \end{enumerate}
\end{frame}

\begin{frame}{Definitions}
    \begin{definition}[Definition 1.3 (iv) Lattice]
        Let $X$ be a partially ordered set. $X$ is a \textbf{lattice} if every subset consisting of two points has a supremum and infimum
    \end{definition}
    \begin{definition}[Definition 4.1 Ordered vector space]
        The real vector space $E$ is called an \textbf{ordered vector space} if $E$ is partially ordered in such a manner that the vector space structure and the order structure are compatible, i.e.,
        \begin{enumerate}
            \item $f\le g\implies f+h\le g+h$ for every $h\in E$
            \item $f\ge 0\implies \alpha f\ge 0$ for every $\alpha \ge 0\in \mathbb{R}$
        \end{enumerate}
    \end{definition}
\end{frame}

\begin{frame}{Definitions}
\begin{definition}[Definition 4.1 Riesz space]
The ordered vector space $E$ is called a \textbf{Riesz space} if $E$ is a lattice with respect to the partial ordering.
\end{definition}

\begin{definition}[p.85 Riesz norm]
    Let $E$ be a Riesz space equipped with a norm. The norm in $E$ is called a \textbf{Riesz norm} if $|f|\le |g|$ in $E$ implies $\|f\|\le \|g\|$.
\end{definition}
\end{frame}

\begin{frame}{Definitions}
    \begin{definition}[p.85 normed Riesz space]
Any Riesz space, equipped with a Riesz norm, is called a \textbf{normed Riesz space}
\end{definition}
\begin{definition}[p.85 Banach lattice]
If the normed Riesz space $E$ is norm complete, $E$ is called a \textbf{Banach lattice}.
\end{definition}
\end{frame}

\begin{frame}{Definitions}
\begin{definition}[Definition 10.9 Directed set]
The non-empty subset $D$ of the Riesz space $E$ is said to be \textbf{upward directed (or downward directed)} if for any two elements $f$ and $g$ in $D$ there exists an element $h\in D$ such that $h\ge f\vee g$ (or $h\le f\wedge g$). 
\end{definition}
\textbf{Notation (p.103)}: If $D$ is an upward (or downwar) directed set in the Riesz space $E$, we write $D\uparrow$ or ($D\downarrow$). If $D\uparrow$ and $D$ has the supremum $f_0$, we write $D\uparrow f_0$. If $D\downarrow$ and $D$ has infimum $f_0$, we write $D\downarrow f_0$.
\end{frame}

\begin{frame}{Definitions}
\begin{definition}[Definition 17.7 Order continuous norm]
The normed Riesz space $E$ is said to have \textbf{order continuous norm} if, for any subset $D\downarrow 0$ in $E$, we have $\inf(\|f\|:f\in D)=0$. The norm is said to be $\sigma$-order continuous if, for any sequence $f_n\downarrow 0$ in $E$, we have $\|f_n\|\downarrow 0$    
\end{definition}
    
\end{frame}

\begin{frame}{Theorem 17.9}
Let $E$ be a Banach lattice and every increasing and bounded above sequence in $E$ converges in norm. Then $E$ has a $\sigma$-order continuous norm.
\begin{proof}
Fix $f_n\downarrow 0$ in $E$. We want to show that $\|f_n\|\downarrow 0$. Let
$$
g_n = f_1-f_n\quad \text{for each $n\ge 1$}
$$
Then $g_n\uparrow$ as $f_n\downarrow$. Moreover, we have for any $n<m$,
$$
g_m-g_n = f_1-f_m-(f_1-f_n)=f_n-f_m\ge 0
$$
Moreover, $g_n$ is bounded above by $f_1$. By our assumption,  this implies $(g_n)$ converges in norm to some element $g\in E$.
\end{proof}
\end{frame}




\end{document}
